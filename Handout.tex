\documentclass{article}
\usepackage[noxcolor]{beamerarticle}
\usepackage{tikz}
\usepackage{pgfpages}
\usepackage{wrapfig}
\usepackage{lipsum}
\usepackage{enumitem}
\usepackage{geometry}

\setjobnamebeamerversion{main}

\setlength\parindent{0pt}

\setenumerate[1]{itemsep=0pt,partopsep=0pt,parsep=0pt,topsep=5pt,leftmargin=\parindent}
\setitemize[1]{itemsep=0pt,partopsep=0pt,parsep=0pt,topsep=5pt,leftmargin=1em}
\setdescription{itemsep=0pt,partopsep=0pt,parsep=0pt,topsep=5pt,leftmargin=\parindent}

\newcommand{\addnote}[2]{
    \begin{minipage}[t]{\textwidth}
        \begin{wrapfigure}{l}{0.5\textwidth}
            \centering
            \vspace{-14pt}
            \includeslide[width = 0.5\textwidth]{#1}
        \end{wrapfigure}\setlength{\parskip}{10pt}
        #2
    \end{minipage}\vfill 
}

\geometry{left=3.5cm,right=3.5cm, top= 3.5cm}
\setlength{\parskip}{20pt}
\begin{document}

\addnote{Title}{
    Hello everyone, my name is Zhang Qi, and I am the Ph.D student of Professor Zhou Chunjie. I am very glad to be invited by Professor Yang Shuanghuang to make a presentation about my recent research.
    
    My research interests are related to risk assessment and decision-making for industrial control systems.     The title of my presentation is ``Multi-Model Based Incident Prediction and Risk Assessment in Dynamic Cybersecurity Protection for Industrial Control Systems''.
}

\addnote{Outlines}{
    My presentation is separated into six parts:
    \begin{itemize}
      \item Firstly, I will introduce the background and the problems of risk assessment for industrial control systems.
      \item Secondly, I will give the architecture of our risk assessment solution for industrial control systems.
      \item Thirdly, I will elaborate the detail of our method.
      \item Then, I will show you the effectiveness of our approach by using a numerical simulation.
      \item At last, I will discuss the problems of our approach and introduce the future works.
    \end{itemize}
}

\addnote{Section: Introduction}{
    In this part, I will introduce the development history and the cybersecurity issues of industrial control systems. And, I will compare the cybersecurity issues of industrial control systems and traditional IT systems.
}

\addnote{Introduction: Development of ICSs}{
    There are four great changes in the development of industrial control systems:
    \begin{itemize}
      \item Machine Age
      \item Semi-automatic Age
      \item Automatic Age
      \item Intelligent Age
    \end{itemize}
    
    The figure shows that with the development of industrial control systems... 
}

\addnote{Introduction: ICSs are Important}{

}

\addnote{Introduction: ICSs are under Attacks}{

}

\addnote{Introduction: Problem of Timeliness and Availability}{

}

\addnote{Introduction: Problem of Overlapping amongst Consequences}{

}

\addnote{Introduction: Problem of Unknown Attacks}{

}

\addnote{Section: Architecture}{

}

\addnote{Architecture: Cybersecurity Risk Assessment for ICSs}{

}

\addnote{Section: Hazardous Incident Prediction}{

}

\addnote{Hazardous Incident Prediction: Attack Level}{

}

\addnote{Hazardous Incident Prediction: Function Tree F1}{

}

\addnote{Hazardous Incident Prediction: Function Tree F2}{

}

\addnote{Hazardous Incident Prediction: Function Tree F5}{

}

\addnote{Hazardous Incident Prediction: Function Tree to Bayesian Network Gate 1}{

}

\addnote{Hazardous Incident Prediction: Function Tree to Bayesian Network Gate 2}{

}

\addnote{Hazardous Incident Prediction: Function Tree to Bayesian Network Gate 3}{

}

\addnote{Hazardous Incident Prediction: Comparison of Fault Tree and Bayesian Network Question}{

}

\addnote{Hazardous Incident Prediction: Comparison of Fault Tree and Bayesian Network Answer}{

}

\addnote{Hazardous Incident Prediction: Incident Level}{

}

\addnote{Hazardous Incident Prediction: Information Transfer from Attack to Function}{

}

\addnote{Hazardous Incident Prediction: Information Transfer from Function to Incident}{

}

\addnote{Hazardous Incident Prediction: Collection of Evidence}{

}

\addnote{Hazardous Incident Prediction: Calculation of Incident Probability}{

}

\addnote{Section: Dynamic Risk Assessment}{

}

\addnote{Dynamic Risk Assessment: Decouple of Incident Consequences Step 1}{

}

\addnote{Dynamic Risk Assessment: Decouple of Incident Consequences Step 2}{

}

\addnote{Dynamic Risk Assessment: Decouple of Incident Consequences Step 3}{

}

\addnote{Dynamic Risk Assessment: Decouple of Incident Consequences Step 4}{

}

\addnote{Dynamic Risk Assessment: Classification of Incident Consequences}{

}

\addnote{Dynamic Risk Assessment: Quantification of Incident Consequences}{

}

\addnote{Dynamic Risk Assessment: Calculation of Dynamic Risk}{

}

\addnote{Section: Simulation}{

}

\addnote{Simulation: Control Structure of Chemical Reactor}{

}

\addnote{Simulation: Structure of Simulation Platform}{

}

\addnote{Simulation: Multi-Level Bayesian Network of Reactor}{

}

\addnote{Simulation: Evidences List}{

}

\addnote{Simulation: Quantification of Consequences}{

}

\addnote{Simulation: Curvers of Cybersecurity Risk and Incident Probability}{

}

\addnote{Simulation: Ability to Deal with the Unknown Attacks}{

}

\addnote{Simulation: Curve of Execution Time Distribution}{

}

\addnote{Simulation: Curves of Risk Assessment Algorithm Scalability}{

}

\addnote{Section: Conclusion and Prospect}{

}

\addnote{Conclusion and Prospect: Conclusion}{

}

\addnote{Conclusion and Prospect: Prospect}{

}

\addnote{Section: Thank You}{

}

\addnote{Thank You: Thank You}{

}

\addnote{Section: Questions}{

}



\end{document}
