\label{Section: Simulation}
\section{Simulation}
\subsection{Simulation Platform}
\begin{frame}{Simulation Platform}
    \label{Simulation: Control Structure of Chemical Reactor}
    The simulation object is a chemical reactor whose control structure is shown as the following figure.\\[-10pt]
    \begin{center}
      \newcommand{\computer}{\includegraphics[width=0.7cm]{Figures/Materials/Computer.pdf}}
\newcommand{\router}{\includegraphics[width=0.7cm]{Figures/Materials/Router.pdf}}
\newcommand{\plc}{\includegraphics[width=0.65cm]{Figures/Materials/PLC.pdf}}
\newcommand{\server}{\includegraphics[width=0.52cm]{Figures/Materials/Server.pdf}}

\newcommand{\mydot}{,}


\newcommand{\bus}[3]{
	\draw[fill = white] (#1, #2 + 0.15) to[out = 180, in = 180] (#1, #2 - 0.15) -- (#1 + #3, #2 - 0.15) to[in = 0, out = 0] (#1 + #3, #2 + 0.15) -- cycle;
	\draw (#1 + #3, #2 + 0.1) to[out = 180, in = 180] (#1 + #3, #2 - 0.1) to[in = 0, out = 0] cycle;
}

\newcommand{\valve}[3]{
	\draw[fill = white] (#1,#2) -- (#1 - 0.25, #2 + 0.15) -- (#1 - 0.25, #2 - 0.15) -- (#1 + 0.25, #2 + 0.15) -- (#1 + 0.25, #2 - 0.15) -- (#1 ,#2) -- (#1, #2 + 0.2) -- (#1 - 0.2, #2 + 0.2) to[out = 45, in = 135] (#1 + 0.2, #2 + 0.2) -- (#1, #2 + 0.2);
	\node (v#3) at (#1,#2) [rectangle, minimum width = 0.3cm] {};
}

\begin{tikzpicture}[line width = \pgfdefaultlinewidth,
                    x = 0.6cm,
                    y = 0.6cm,
                    tag/.style = {pos = 0.7}]
\scriptsize

% Networks
\draw (2,9) node {\server} -- (2,8);
\draw (5,9) node {\computer} -- (5,8);
\draw (8,9.5) -- (8,9) node {\router} -- (8,8);
\draw (3.5,8) -- (3.5,7) node {\router} -- (3.5,6);
\draw (8,8) -- (8,7) node {\router} -- (8,6);

\foreach \i in {1,...,6}{
    \draw (1.5*\i + 0.5,6) -- (1.5*\i + 0.5,5) node {\plc} -- (1.5*\i + 0.5,4);
    \node [below = -8pt, text shadow={[align=center, below = -8pt] at (1.5*\i + 0.5,4) {PLC\i}}] at (1.5*\i + 0.5,4) {PLC\i};
}

% Water Level
\fill[blue!50] (4,-1) to[out = -90, in = -90] (7.5,-1) -- (7.5, 1) decorate [decoration={snake, segment length = 4mm, amplitude = 0.5mm}] { -- (4, 1)};

% Tank
\draw (4,2) -- (4,-1) to[out = -90, in = -90] (7.5,-1) -- (7.5,2) to [in = 90, out = 90] cycle;

% Sensor
\draw[line width = 3pt] (4.4, 0.3) -- (4.4, 0.7);
\draw[line width = 3pt] (4.85, 0.3) -- (4.85, 0.7);
\draw[line width = 3pt] (5.3, 0.3) -- (5.3, 0.7);

\foreach \i in {4.4, 4.85, 5.3, 5.75}{
	\draw[line width = 2pt, white] (\i,3.1) -- (\i,2);
}

\draw(4.4,0.7) -- (4.4, 3.2);
\draw(4.85,0.7) -- (4.85, 3.2);
\draw(5.3,0.7) -- (5.3, 3.2);

% Link
\node (M) at (5.75, 3.5) [circle, draw, inner sep = 1pt] {M};
\draw (2,4) -- (2,1) -- (2.75,1) -- (2.75,0.5);
\draw (2,2) -- (2.75,2) -- (2.75,1.5);
\draw (3.5,4) -- (3.5, 3.2) -- (5.3, 3.2);
\draw (5,4) -- (5,3.5) -- (M);
\draw (6.5,4) -- (6.5,3.2) -- (8,3.2) -- (8,-0.45);
\draw (8,4) -- (8,3.5) -- (8.75, 3.5) -- (8.75,1.5);
\draw (9.5,4) -- (9.5,1) -- (8.75,1) -- (8.75,0.5);

\draw[fill = black] (4.4,3.2) circle (0.2mm);
\draw[fill = black] (4.85,3.2) circle (0.2mm);
\draw[fill = black] (5.3,3.2) circle (0.2mm);
\draw[fill = black] (2,2) circle (0.2mm);

% Valve
\valve{2.75}{1.5}{1}
\valve{2.75}{0.5}{2}
\valve{8.75}{0.5}{3}
\valve{8.75}{1.5}{4}

\draw[line width = 4pt, white] (1.5, 1.5) -- (v1);
\draw[line width = 4pt, white] (7.6, 0.5) -- (v3);
\draw[line width = 4pt, white] (7.6,1.5) -- (v4) -- (10, 1.5);
\draw[line width = 2pt, ->] (1.5, 1.5) -- (v1) -- (  4, 1.5);
\draw[line width = 2pt, ->] (1.5, 0.5) -- (v2) -- (  4, 0.5);
\draw[line width = 2pt, ->] (7.5, 0.5) -- (v3) -- ( 10, 0.5);
\draw[line width = 2pt, ->] (7.5, 1.5) -- (v4) -- ( 10, 1.5);

% Blender
\draw[fill = black] (M) -- (5.75, 0) to[out = 10, in = 90] (6.75, 0) to[out = -90, in = -10] (5.75, 0) to[out = 170, in = 90] (4.75,0) to[out = -90, in = 190] (5.75, 0);

% Heater
\node (AC) at (9.5, -1) [draw, circle, inner sep = 1pt]{$\sim$};
\node (SW) at (  8, -0.5) [rectangle, minimum size = 0.3cm]{};

% Switch
\draw (AC) -- (9.5, -0.5) -- (SW) -- (5.75 ,-0.5) decorate [decoration={bumps, segment length = 1.5mm, amplitude = 1.5mm, post length = 0.5mm, pre length = 0.5mm}] { -- (5.75, -1.5)} -- (9.5, -1.5) -- (AC);
\draw[fill = white] (8.25, -0.4) -- (7.75, -0.5) circle (0.3mm);

% Bus
\bus{1.5}{8}{8.5}
\bus{1.5}{6}{4}
\bus{6}{6}{4}

\node at (5.75,8) {\tiny Ethernet};
\node at ( 3.5,6) {\tiny CANBUS};
\node at (   8,6) {\tiny CANBUS};

% Tag
\node at (v1) [below right=-10pt and 1pt]{V1};
\node at (v2) [below right=-10pt and 1pt]{V2};
\node at (v3) [below right=-10pt and 1pt]{V3};
\node at (v4) [below right=-10pt and 1pt]{V4};
\node at (5.75,0)[anchor = north, text shadow={[align=center, anchor = north] at (5.75,0) {B}}]{B};
\node at (5.75,-1)[anchor = east, text shadow={[align=center, anchor = east] at (5.75,-1) {H}}]{H};
\node at (SW) [anchor = north]{SW};
\node at (4.4,1) [text shadow={[align=center,text width=3cm] at (4.4,1) {T}}] {T};
\node at (4.85,1) [text shadow={[align=center,text width=3cm] at (4.85,1) {P}}] {P};
\node at (5.3,1) [text shadow={[align=center,text width=3cm] at (5.3,1) {L}}] {L};
\node at (2.3,9) [right = 2pt] {HDS};
\node at (5.4,9) [right = 2pt] {ES};
\node at (8.4,9) [right = 2pt] {G1};
\node at (3.9,7) [right = 2pt] {G2};
\node at (8.4,7) [right = 2pt] {G3};

% Legend
\fill[black] (10.3 + 0.5, -2) rectangle (17.8 + 0.5,9.3);
\draw[fill = white] (10.2 + 0.5,-1.9) rectangle (17.7 + 0.5,9.4);

\foreach \i/\s/\t in {1/HDS/Historical data server,
                      2/ES/Engineer station,
                      3/G1/Gateway of Ethernet,
                      4/G2/Gateway of CANBUS,
                      5/G3/Gateway of CANBUS,
                      6/PLC1/Controller of V1 and V2,
                      7/PLC2/Data collection of P\mydot{} T and L,
                      8/PLC3/Controller of M,
                      9/PLC4/Controller of SW,
                      10/PLC5/Controller of V4,
                      11/PLC6/Controller of V3,
                      12/V1/Valve of material,
                      13/V2/Valve of another material,
                      14/V3/Valve of product,
                      15/V4/Valve of pressure reducing,
                      16/M/Motor of B,
                      17/SW/Switch of H,
                      18/P/Pressure sensor,
                      19/T/Temperature sensor,
                      20/L/Liquid level sensor,
                      21/B/Blender,
                      22/H/Heater}
{
	\draw (10.5 + 0.5, 9.4 - 0.5*\i) node [anchor = west] {\s};
	\draw (11.9 + 0.5, 9.4 - 0.5*\i) node [anchor = west] {\t};
}

\node at (11,9.4) [fill = white, anchor = west] {Legend};

\end{tikzpicture}
    \end{center}
\end{frame}

\begin{frame}{Simulation Platform}
    \label{Simulation: Structure of Simulation Platform}
    The simulation platform is implemented in Matlab, which consists of three modules: an evidence generator, an incident prediction module, and a risk assessment module.
    \begin{center}
      \newcommand{\myscope}[2]
{
	\draw (#1, #2 - 0.7) rectangle (#1 + 1, #2 + 0.7);
	\draw (#1 + 0.1, #2 + 0.1) rectangle (#1 + 0.9, #2 + 0.6);
	\node at (#1 + 0.5, #2 + 0.9) {Scope};
}

\begin{tikzpicture}[line width = \pgfdefaultlinewidth,
                    x = 0.5cm,
                    y = 0.5cm,
					block/.style = {rectangle, draw, minimum width = 2cm, minimum height = 2.75cm, inner sep = 0pt, align = center},
                    attack/.style =   {draw = red,       fill = red,       circle, minimum size = 0.12cm, inner sep=0pt},
					resource/.style = {draw = green,     fill = green,     circle, minimum size = 0.12cm, inner sep=0pt},
					function/.style = {draw = blue,      fill = blue,      circle, minimum size = 0.12cm, inner sep=0pt},
					incident/.style = {draw = black,     fill = black,     circle, minimum size = 0.12cm, inner sep=0pt}]

% \grid{0}{22}{0}{10}

\fontsize{5pt}{5pt}\selectfont

\node[block] (ES)   at (3.5, 5.75) {};
\node[block, minimum width = 2.5cm] (MLBN) at (9.5, 5.75) {};
\node[block, minimum width = 2.5cm, anchor = west] (RA)   at (15, 5.75) {};

\node at (3.5, 8) {\tiny Evidence Generator};
\node at (9.5, 8) {\tiny Incident Prediction Module};
\node at (17.5, 8) {\tiny Risk Assessment Module};

\node at (9.5, 5.5) {\resizebox{2.4cm}{!}{\begin{tikzpicture}[line width = \pgfdefaultlinewidth,
					attack/.style =   {draw = red,       fill = red,       circle, minimum size = 0.25cm, inner sep=0pt},
					resource/.style = {draw = green,     fill = green,     circle, minimum size = 0.25cm, inner sep=0pt},
					function/.style = {draw = blue,      fill = blue,      circle, minimum size = 0.25cm, inner sep=0pt},
					incident/.style = {draw = black,     fill = black,     circle, minimum size = 0.25cm, inner sep=0pt}]

\pgfmathsetmacro{\interval}{1.2}

\node[attack]    (a1) at (12,  1*\interval) {};
\node[attack]    (a2) at ( 8,  1*\interval) {};
\node[attack]    (a6) at (16,  1*\interval) {};

\foreach \i/\x in {2/6,
				   3/8,
				   4/10,
				   1/12,
				   6/14,
				   5/16,
				   8/18}{
	\node[resource] (r\i) at (\x, 2*\interval) {};
}

\foreach \i/\x in {3/7,
				   4/9,
				   5/11,
				   8/13,
				   7/15,
				   9/17}{
	\node[attack] (a\i) at (\x, 3*\interval) {};
}

\foreach \i/\x in {7/9,
				   9/15}{
	\node[resource] (r\i) at (\x, 4*\interval) {};
}

\foreach \i in {10,...,27}{
	\node[attack] (a\i) at (\i - 6.5, 7*\interval) {};
}

\foreach \i in {1,...,9}{
	\node[function] (f\i) at (2 + 2*\i, 10*\interval) {};
}

\foreach \i/\x in {10/7,
				   11/12,
				   12/17}{
	\node[function] (f\i) at (\x, 11*\interval) {};
}

\foreach \i/\x in {1/6,
				   2/9,
				   3/12,
				   4/15,
				   8/18}{
	\node[incident] (e\i) at (\x, 12*\interval) {};
}

\foreach \i/\x in {5/9,
				   7/12,
				   6/15}{
	\node[incident] (e\i) at (\x, 13*\interval) {};
}

\foreach \i/\x in {11/5,
				   15/7,
				   14/9,
				   12/11,
				   16/13,
				   13/15,
				   9/17,
				   10/19}{
	\node[incident] (e\i) at (\x, 14*\interval) {};
}


\foreach \i/\j in{a1/r1,
                  r1/a2,
                  r1/a6,
                  r1/a3,
                  r1/a4,
                  r1/a5,
                  r1/a7,
                  r1/a8,
                  r1/a9,
                  a2/r2,
                  a2/r3,
                  a2/r4,
                  a2/r5,
                  a2/r6,
                  r2/a3,
                  r3/a4,
                  r4/a5,
                  r5/a7,
                  r6/a8,
                  r6/a9,
                  a6/r8,
                  r8/a9,
                  a3/r7,
                  a4/r7,
                  a5/r7,
                  a7/r9,
                  r7/a8,
                  r7/a10,
                  r7/a11,
                  r7/a12,
                  r7/a13,
                  r7/a14,
                  r7/a15,
                  r7/a22,
                  r7/a23,
                  r7/a24,
                  r7/a25,
                  r7/a26,
                  r7/a27,
                  a8/r9,
                  a9/r9,
                  r9/a10,
                  r9/a11,
                  r9/a12,
                  r9/a13,
                  r9/a14,
                  r9/a15,
                  r9/a16,
                  r9/a17,
                  r9/a18,
                  r9/a19,
                  r9/a20,
                  r9/a21,
                  r9/a22,
                  r9/a23,
                  r9/a24,
                  r9/a25,
                  r9/a26,
                  r9/a27,
                  a10/f1,
                  a10/f2,
                  a11/f7,
                  a11/f8,
                  a11/f9,
                  a12/f6,
                  a13/f5,
                  a14/f4,
                  a15/f3,
                  a16/f1,
                  a16/f2,
                  a17/f7,
                  a17/f8,
                  a17/f9,
                  a18/f6,
                  a19/f5,
                  a20/f4,
                  a21/f3,
                  a22/f1,
                  a22/f2,
                  a23/f7,
                  a23/f8,
                  a23/f9,
                  a24/f6,
                  a25/f5,
                  a26/f4,
                  a27/f3,
                  f1/f10,
                  f1/f11,
                  f2/f10,
                  f2/f11,
                  f3/f10,
                  f3/f11,
                  f4/f12,
                  f5/f11,
                  f6/e8,
                  f7/f10,
                  f8/f11,
                  f9/f12,
                  f10/e1,
                  f10/e2,
                  f11/e3,
                  f12/e4,
                  e1/e5,
                  e2/e7,
                  e3/e9,
                  e4/e6,
                  e4/e9,
                  e4/e10,
                  e5/e11,
                  e6/e9,
                  e6/e10,
                  e6/e11,
                  e6/e12,
                  e6/e13,
                  e6/e14,
                  e6/e15,
                  e6/e16,
                  e7/e9,
                  e7/e14,
                  e7/e15,
                  e8/e9}{
	\draw[->] (\i) -- (\j);
}

\end{tikzpicture} }};
\node at (3.5, 5.5) {\resizebox{1.9cm}{!}{\input{Figures/Simulation/EvidenceListMini}}};


\draw[-{>[scale = 0.5, length=5, width = 6]}] (0.2,5.75) -- (1.5,5.75) node [midway, above] {Trigger};
\draw[{<[scale = 0.5, length=5, width = 6]}-{>[scale = 0.5, length=5, width = 6]}, line width = 0.5pt] (0.4, 4.9) -- (1.2, 4.9) node [midway, fill = white, rotate = 45, inner sep = 1pt] {\fontsize{4pt}{4pt}\selectfont 1 min};
\draw[line width = 0.5pt] (0.2,5.6) -- (0.4, 5.6) -- (0.4,5.2) -- (0.8,5.2) -- (0.8,5.6) -- (1.2, 5.6) -- (1.2,5.2) -- (1.4,5.2);
\draw[line width = 0.5pt] (0.4, 4.7) -- (0.4, 5.1);
\draw[line width = 0.5pt] (1.2, 4.7) -- (1.2, 5.1);

\draw[-{>[scale = 0.5, length=5, width = 6]}] (5.5, 6.5) -- (7, 6.5) node [midway, sloped, align = center] {Attack\\ Evidence};

\draw[-{>[scale = 0.5, length=5, width = 6]}] (5.5, 5) -- (7, 5) node [midway, sloped, align = center] {Anomaly\\ Evidence};

\draw[white, line width = 3pt] (19.6, 5.75) -- (20.4,5.75);
\foreach \y/\t in {1/x_1,
				   2/x_2,
				   3/x_3,
				   4/x_4,
				   5/x_5,
				   6/x_6,
				   7/x_7,
				   8/x_8}{
	\node (t\y) at (15 + 0.5*\y, 8-0.5*\y) [circle, fill = black, inner sep = 0pt, minimum size = 0.15cm] {\textcolor[rgb]{1,1,1}{\scalebox{0.5}{$\bm{\times}$}}};
	\draw[white, line width = 1.5pt] (13, 8 - 0.5*\y) -- (t\y);
	\fill (13 + 0.2*\y, 8 - 0.5*\y) circle (0.05cm);
	\draw[-{>[scale = 0.5, length=5, width = 6]}] (13 + 0.2*\y, 8 - 0.5*\y) -- (13 + 0.2*\y, 0.7 + 0.2*\y) -- (15.5, 0.7 0+ 0.2*\y);
	
	\node at (12, 8 - 0.5*\y - 0.1) [anchor = south west] {$p(\t)$};
		
	\draw[-{>[scale = 0.5, length=5, width = 6]}] (12, 8 - 0.5*\y) -- (t\y);
	\draw[-{>[scale = 0.5, length=5, width = 6]}] (t\y) -- (19.4, 8 - 0.5*\y);
	\draw[-{>[scale = 0.5, length=5, width = 6]}] (15 + 0.5*\y, 3.6) -- (t\y);
	\node at (15 + 0.5*\y, 3.45) [rotate = 45] {$q(\t)$};
	
	\fill (19.4,3.8) rectangle (19.8,7.7) ;
}


\draw[-{>[scale = 0.5, length=5, width = 6]}] (19.5, 5.75) -- (20.8,5.75) node [below left= -7pt and -1pt] {Risk};
\node at (19.6,5.75) {\textcolor[rgb]{1,1,1}{$\sum$}};

\fill[black] (15.5, 0.7) rectangle (15.7,2.5);
\node at (15.6, 2.7) {Mux};

\draw[-{>[scale = 0.5, length=5, width = 6]}] (15.7, 1.6) -- (16.5, 1.6);

\myscope{16.5}{1.6}
\myscope{20.8}{5.75}

\foreach \x/\i/\e/\t in {0/1/x_{1}/Product damaged,
					     0/2/x_{2}/Tank damaged,
					     0/3/x_{3}/Heater damaged,
					     0/4/x_{4}/Sensors damaged,
					     1/1/x_{5}/Staff$_{1\text{-}4}$ injured,
					     1/2/x_{6}/Staff$_{5\text{-}9}$ injured,
					     1/3/x_{7}/Water pollution,
					     1/4/x_{8}/Air pollution}{
	\node at (5.2 + 4*\x + 0.7, 2.9 - 0.5*\i) [anchor = east] {$\e$};
	\node at (5.2 + 4*\x + 0.5, 2.9 - 0.5*\i) [anchor = west] {-- \t};
}

\foreach \i/\c/\t in {1/attack/Attack node,
					  2/resource/Resource node,
					  3/function/Function node,
					  4/incident/Incident node}{
	%\fill[\c] (2,2.9 - 0.5*\i) circle (0.12cm);
    \node[\c] at (1.7,2.9 - 0.5*\i){};
	\node at (1.9,2.9 - 0.5*\i) [anchor = west] {\t};
}


\end{tikzpicture}

    \end{center}
\end{frame}

\begin{frame}{Simulation Platform}
    \label{Simulation: Multi-Level Bayesian Network of Reactor}
    The multi-level Bayesian network of the chemical reactor is shown as following figure.\vspace{-18pt}\\
    \begin{center}
      \begin{tikzpicture}[line width = \pgfdefaultlinewidth,
                    x = 0.25cm,
                    y = 0.25cm,
					attack/.style = {draw = red, fill = red, text = white, circle, minimum size = 0.2cm, inner sep=0pt},
					resource/.style = {draw = green, fill = green, text = white, circle, minimum size = 0.2cm, inner sep=0pt},
					function/.style = {draw = blue, fill = blue, text = white, circle, minimum size = 0.2cm, inner sep=0pt},
					incident/.style = {draw = black, circle, fill = black, text = white, minimum size = 0.2cm, inner sep=0pt},
					auxiliary/.style = {draw = black, circle, fill = black, text = white, minimum size = 0.2cm, inner sep=0pt},
                    arrow/.style = {-{>[scale = 0.5, length=4, width = 6]}}]

\fontsize{3pt}{3pt}\selectfont

\node[attack]    (a1) at (12,  1) {$a_1$};
\node[attack]    (a2) at ( 8,  1) {$a_2$};
\node[attack]    (a6) at (16,  1) {$a_6$};

\foreach \i/\x in {2/6,
				   3/8,
				   4/10,
				   1/12,
				   6/14,
				   5/16,
				   8/18}{
	\node[resource] (r\i) at (\x, 3) {$r_{\i}$};
}

\foreach \i/\x in {3/7,
				   4/9,
				   5/11,
				   8/13,
				   7/15,
				   9/17}{
	\node[attack] (a\i) at (\x, 5) {$a_{\i}$};
}

\foreach \i/\x in {7/9,
				   9/15}{
	\node[resource] (r\i) at (\x, 7) {$r_{\i}$};
}

\foreach \i in {10,...,27}{
	\node[attack] (a\i) at (\i - 6.5, 13) {$a_{\i}$};
}

\foreach \i in {1,...,9}{
	\node[function] (f\i) at (2 + 2*\i, 19) {$f_{\i}$};
}

\foreach \i/\x in {10/7,
				   11/12,
				   12/17}{
	\node[function] (f\i) at (\x, 22) {$f_{\i}$};
}

\foreach \i/\x in {1/6,
				   2/9,
				   3/12,
				   4/15,
				   8/18}{
	\node[incident] (e\i) at (\x, 24) {$e_{\i}$};
}

\foreach \i/\x in {5/9,
				   7/12,
				   6/15}{
	\node[incident] (e\i) at (\x, 26) {$e_{\i}$};
}

\foreach \i/\x in {3/5,
				   7/7,
				   6/9,
				   4/11,
				   8/13,
				   5/15,
				   1/17,
				   2/19}{
	\node[incident] (x\i) at (\x, 29) {$x_{\i}$};
}


\foreach \i/\j in{a1/r1,
                  r1/a2,
                  r1/a6,
                  r1/a3,
                  r1/a4,
                  r1/a5,
                  r1/a7,
                  r1/a8,
                  r1/a9,
                  a2/r2,
                  a2/r3,
                  a2/r4,
                  a2/r5,
                  a2/r6,
                  r2/a3,
                  r3/a4,
                  r4/a5,
                  r5/a7,
                  r6/a8,
                  r6/a9,
                  a6/r8,
                  r8/a9,
                  a3/r7,
                  a4/r7,
                  a5/r7,
                  a7/r9,
                  r7/a8,
                  r7/a10,
                  r7/a11,
                  r7/a12,
                  r7/a13,
                  r7/a14,
                  r7/a15,
                  r7/a22,
                  r7/a23,
                  r7/a24,
                  r7/a25,
                  r7/a26,
                  r7/a27,
                  a8/r9,
                  a9/r9,
                  r9/a10,
                  r9/a11,
                  r9/a12,
                  r9/a13,
                  r9/a14,
                  r9/a15,
                  r9/a16,
                  r9/a17,
                  r9/a18,
                  r9/a19,
                  r9/a20,
                  r9/a21,
                  r9/a22,
                  r9/a23,
                  r9/a24,
                  r9/a25,
                  r9/a26,
                  r9/a27,
                  a10/f1,
                  a10/f2,
                  a11/f7,
                  a11/f8,
                  a11/f9,
                  a12/f6,
                  a13/f5,
                  a14/f4,
                  a15/f3,
                  a16/f1,
                  a16/f2,
                  a17/f7,
                  a17/f8,
                  a17/f9,
                  a18/f6,
                  a19/f5,
                  a20/f4,
                  a21/f3,
                  a22/f1,
                  a22/f2,
                  a23/f7,
                  a23/f8,
                  a23/f9,
                  a24/f6,
                  a25/f5,
                  a26/f4,
                  a27/f3,
                  f1/f10,
                  f1/f11,
                  f2/f10,
                  f2/f11,
                  f3/f10,
                  f3/f11,
                  f4/f12,
                  f5/f11,
                  f6/e8,
                  f7/f10,
                  f8/f11,
                  f9/f12,
                  f10/e1,
                  f10/e2,
                  f11/e3,
                  f12/e4,
                  e1/e5,
                  e2/e7,
                  e3/x1,
                  e4/e6,
                  e4/x1,
                  e4/x2,
                  e5/x3,
                  e6/x1,
                  e6/x2,
                  e6/x3,
                  e6/x4,
                  e6/x5,
                  e6/x6,
                  e6/x7,
                  e6/x8,
                  e7/x1,
                  e7/x6,
                  e7/x7,
                  e8/x1}{
	\draw[arrow] (\i) -- (\j);
}

\foreach \i/\c/\n/\e in { 1/1/a_{ 1}/Network Scanning,
                          2/1/a_{ 2}/Vulnerability scanning,
                          3/1/a_{ 3}/Buffer overflow attack on HDS,
                          4/1/a_{ 4}/FTP attack on HDS,
                          5/1/a_{ 5}/Brute force attack on HDS,
                          6/1/a_{ 6}/DoS attack on HDS ,
                          7/1/a_{ 7}/Buffer overflow attack on ES,
                          8/1/a_{ 8}/Privilege escalation attack on ES,
                          9/1/a_{ 9}/Spoofing attack on ES,
                         10/1/a_{10}/DoS attack on PLC1,
                         11/1/a_{11}/DoS attack on PLC2,
                         12/1/a_{12}/DoS attack on PLC3,
                         13/1/a_{13}/DoS attack on PLC4,
                         14/1/a_{14}/DoS attack on PLC5,
                         15/1/a_{15}/DoS attack on PLC6,
                         16/1/a_{16}/Reconfigure PLC1,
                         17/1/a_{17}/Reconfigure PLC2,
                         18/1/a_{18}/Reconfigure PLC3,
                         19/1/a_{19}/Reconfigure PLC4,
                         20/1/a_{20}/Reconfigure PLC5,
                         21/1/a_{21}/Reconfigure PLC6,
                         22/1/a_{22}/Man-in-the-middle attack on PLC1,
                         23/1/a_{23}/Man-in-the-middle attack on PLC2,
                         24/1/a_{24}/Man-in-the-middle attack on PLC3,
                         25/1/a_{25}/Man-in-the-middle attack on PLC4,
                         26/1/a_{26}/Man-in-the-middle attack on PLC5,
                         27/1/a_{27}/Man-in-the-middle attack on PLC6,
                         28/1/r_{ 1}/IP addresses of HDS and ES,
                         29/1/r_{ 2}/Buffer overflow vulnerability,
                         30/1/r_{ 3}/FTP server vulnerability,
                         31/1/r_{ 4}/Login vulnerability,
                         32/1/r_{ 5}/Buffer overflow vulnerability,
                         33/2/r_{ 6}/Authentication vulnerability,
                         34/2/r_{ 7}/Administrator authority of HDS,
                         35/2/r_{ 8}/Crash of HDS,
                         36/2/r_{ 9}/Administrator authority of ES,
                         37/2/f_{ 1}/Traffic control of V1,
                         38/2/f_{ 2}/Traffic control of V2,
                         39/2/f_{ 3}/Traffic control of V3,
                         40/2/f_{ 4}/Pressure reducing,
                         41/2/f_{ 5}/Heating function,
                         42/2/f_{ 6}/Mixing function,
                         43/2/f_{ 7}/Liquid level sensation,
                         44/2/f_{ 8}/Temperature sensation,
                         45/2/f_{ 9}/Pressure sensation,
                         46/2/f_{10}/Liquid level control,
                         47/2/f_{11}/Temperature control,
                         48/2/f_{12}/Pressure control,
                         49/2/e_{ 1}/Excessive liquid level,
                         50/2/e_{ 2}/Low liquid level,
                         51/2/e_{ 3}/Temperature anomaly,
                         52/2/e_{ 4}/Excessive pressure,
                         53/2/e_{ 5}/Heater dry fired,
                         54/2/e_{ 6}/Reactor explosion,
                         55/2/e_{ 7}/Liquid overflow,
                         56/2/e_{ 8}/Blender stop,
                         57/2/x_{ 1}/Production damaged,
                         58/2/x_{ 2}/Tank damaged,
                         59/2/x_{ 3}/Heater damaged,
                         60/2/x_{ 4}/Sensors damaged,
                         61/2/x_{ 5}/Staff$_{\text{1-4}}$ injured,
                         62/2/x_{ 6}/Staff$_{\text{5-9}}$ injured,
                         63/2/x_{ 7}/Water pollution,
                         64/2/x_{ 8}/Air pollution}{
	\node[anchor = east] at (9.2 + 14*\c, 1 - 0.9*\i + 28.8*\c) {\tiny $\n$ \ -- };
	\node[anchor = west] at (9.2 + 14*\c, 1 - 0.9*\i + 28.8*\c) {\tiny \e};
}
\end{tikzpicture}

    \end{center}
\end{frame}

\begin{frame}{Simulation Platform}
    \label{Simulation: Evidences List}
    The list of evidences is shown as following table.

    \taburowcolors[2] 2{black!10 .. black!30}
    \extrarowsep = 1mm
    \begin{tabu} to \textwidth{*2{X[c,-1]|[1.5pt white]}X[l]|[1.5pt white]X[-1, c]}
    \rowcolor{black!80}\rowfont\bfseries \textcolor{white}{Start}  & \textcolor{white}{End} & \textcolor{white}{Description} & \textcolor{white}{Symbol}\\\tabucline[1.5pt white]{-}
    50            & 60        & IP sweep                         & $L(a_1)$ \\\tabucline[1.5pt white]{-}
    75            & 110       & Vulnerability scanning           & $L(a_2)$ \\\tabucline[1.5pt white]{-}
    120           & 180       & DoS attack to HDS                & $L(a_6)$ \\\tabucline[1.5pt white]{-}
    157           & 171       & IP address spoofing              & $L(a_9)$ \\\tabucline[1.5pt white]{-}
    259           & 261       & Reconfigure PLC5                 & $L(a_{20})$ \\\tabucline[1.5pt white]{-}
    266           & 378       & Switch function of V4 failed     & $F(f_4)$ \\\tabucline[1.5pt white]{-}
    286           & 390       & Pressure reduce function failed  & $F(f_{12})$ \\\tabucline[1.5pt white]{-}
    310           & 400       & Pressure is excessive            & $H(e_4)$ \\

    \end{tabu}
\end{frame}

\begin{frame}{Simulation Platform}
    \label{Simulation: Quantification of Consequences}
    The quantification of consequences is shown as following table.

    \newcolumntype Z{X[m, c, 1.8]{%
    S[
    %group-four-digits=true ,
    round-mode = figures,
    group-separator = {,},
    add-decimal-zero = false,
    round-integer-to-decimal=true ,
    table-number-alignment = center,
    table-space-text-pre = \hspace{30pt},
    per-mode=symbol]}}
    \tabucolumn Z
    \taburowcolors[2] 2{black!10 .. black!30}
    \tabulinesep = 1mm
    \extrarowsep = 1mm
    \begin{tabu}to \textwidth{X[c,1, m]|[1.5pt white]X[m, 3]|[1.5pt white]Z}

    \rowcolor{black!80}\rowfont\bfseries
        \textcolor{white}{Incident Symbol} &
        \textcolor{white}{Description of Incident} &
        \textcolor{white}{\bf Quantification of Consequence(\$)}\\
    \tabucline[1.5pt white]{-}
        $x_{1}$  & Product damaged              & 50000  \\\tabucline[1.5pt white]{-}
        $x_{2}$  & Tank damaged                 & 500000 \\\tabucline[1.5pt white]{-}
        $x_{3}$  & Heater damaged               & 10000  \\\tabucline[1.5pt white]{-}
        $x_{4}$  & Sensors damaged              & 10000  \\\tabucline[1.5pt white]{-}
        $x_{5}$  & Staff$_{\text{1-4}}$ injured & 800000 \\\tabucline[1.5pt white]{-}
        $x_{6}$  & Staff$_{\text{5-9}}$ injured & 1000000\\\tabucline[1.5pt white]{-}
        $x_{7}$  & Water pollution              & 200000 \\\tabucline[1.5pt white]{-}
        $x_{8}$  & Air pollution                & 200000 \\
    \end{tabu}
\end{frame}

\subsection{Simulation and Result Analysis}
\begin{frame}{Simulation and Result Analysis}
    \label{Simulation: Curvers of Cybersecurity Risk and Incident Probability}
    \begin{center}
      \input{Figures/Simulation/Curver.of.Cybersecurity.Risk.tex}\\
      \input{Figures/Simulation/Curvers.of.Incident.Probability.tex}\\
    \end{center}
\end{frame}

\begin{frame}{Simulation and Result Analysis}
    \label<trans:1>{Simulation: Ability to Deal with the Unknown Attacks Step 1}
    \label<trans:2>{Simulation: Ability to Deal with the Unknown Attacks Step 2}
    \label<trans:3>{Simulation: Ability to Deal with the Unknown Attacks Step 3}
    \vspace{10pt}
    \begin{overlayarea}{\textwidth}{1cm}
        \only<1| trans:1>{In the previous simulation, the curve of the cybersecurity risk is shown as the \textcolor{red}{\bf red} line in the following figure.}
        \only<2| trans:2>{To validate the ability to deal with the unknown attacks, the attack knowledge about attack \textcolor{red}{\st{$a_6$}} and attack \textcolor{red}{\st{$a_9$}} is removed from the multi-level Bayesian network.}
        \only<3| trans:3>{Then an identical multi-step attack on the system is launched to the system. The new cybersecurity risk curve is shown the dashed line in the following figure.}
    \end{overlayarea}\vspace{10pt}
    \begin{center}
      \input{Figures/Simulation/Curvers.of.Cybersecurity.Risk.tex}
    \end{center}
\end{frame}

\begin{frame}{Simulation and Result Analysis}
    \label<trans:1>{Simulation: Curve of Execution Time Distribution Step 1}
    \label<trans:2>{Simulation: Curve of Execution Time Distribution Step 2}
    \label{Simulation: Curve of Execution Time Distribution}
    \vspace{10pt}
    \begin{overlayarea}{\textwidth}{1.8cm}
        \only<1| trans:1>{We repeat the first simulation 5,000 times, and the execution time of 5,000 calculations is recorded. This simulation is run on a machine with Intel Pentium processor G3220 (3M Cache, 3.00GHz) and 4GB DDR3 memory. The following figure shows the distribution of the 5,000 execution times.}
        \only<2-| trans:2>{Some parameters of the following figure:
        \begin{itemize}
           \item<2-> The average execution time of a risk assessment is $94.1$ms.
           \item<3-> The minimum execution time of a risk assessment is $89.9$ms.
           \item<4-> The maximum execution time of a risk assessment is $131.6$ms.
         \end{itemize}}
    \end{overlayarea}\vspace{10pt}
    \begin{center}
      \input{Figures/Simulation/Curve.of.Execution.Time.Distribution.tex}
    \end{center}
\end{frame}

\begin{frame}{Simulation and Result Analysis}
    \label<trans:1>{Simulation: Curves of Risk Assessment Algorithm Scalability Step 1}
    \label<trans:2>{Simulation: Curves of Risk Assessment Algorithm Scalability Step 2}
    \label<trans:3>{Simulation: Curves of Risk Assessment Algorithm Scalability Step 3}
    \vspace{10pt}
    \begin{overlayarea}{\textwidth}{1.8cm}
    \only<1| trans:1>{Finally, 25 multi-level Bayesian networks with different node sizes are adopted to show the possible upper/lower bounds and the scalability of our approach.}
    \only<2| trans:2>{In the following figure, the fitting line $$y = 0.0019x - 0.0175$$ matches well with the correlation coefficient $r = 0.9987$.}
    \only<3| trans:3>{This means that the execution time of the risk assessment scales linearly with the increase of the node size of the multi-level Bayesian network.}
    \end{overlayarea}\vspace{10pt}
    \begin{center}
      \input{Figures/Simulation/Curves.of.Risk.Assessment.Algorithm.Scalability}
    \end{center}
\end{frame}

