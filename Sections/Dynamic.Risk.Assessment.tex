\section{Dynamic Risk Assessment}
\subsection{Decouple of Incident Consequences}
\begin{frame}{Decouple of Incident Consequences -- Step 1}
     for each incident $e_i$, analyze its consequence and generate a consequence set
     \[
        \bm{c}_i = (c_1, c_2, \cdots, c_n) \text{.}
     \]

     The meaning of $\bm{c}_i$ is that the occurring of the incident $e_i$ will threaten the elements in consequence set $\bm{c}_i$.\\[10pt]

     For example, the incident $e_i$ is an explosion of a reactor, which may cause worker casualties, air pollution, facilities damages, and products loss. The consequence set of $e_i$ is
     \[
        \bm{c}_i = (\text{workers}, \text{air}, \text{facilities}, \text{products})\text{.}
     \]
\end{frame}

%\begin{frame}{Decouple of Incident Consequences -- Step 2}
%    Generate $\bm{C}' = (\bm{c}'_1, \bm{c}'_2, \cdots, \bm{c}'_{m'})$ based on $\bm{C} = (\bm{c}_1, \bm{c}_2, \cdots, \bm{c}_m)$. The following conditions must be met:\vspace{-10pt}
%
%    \[
%        \begin{array}{rl}
%            \text{\bf Completeness: } & \bigcup_{i=1}^m \bm{c}_i = \bigcup_{i=1}^{m'}\bm{c}'_i\\[5pt]
%            \text{\bf Independence: } & \forall \bm{c}'_i, \bm{c}'_j \in \bm{C}', \bm{c}'_i \cap \bm{c}'_j = \varnothing\\[5pt]
%            \text{\bf Traceability: } & \forall \bm{c}' \in \bm{C}', \exists \bm{c} \in \bm{C}, \bm{c}' \subseteq \bm{c}
%        \end{array}
%    \]
%
%    \pause
%    \vspace{-10pt}
%    \begin{center}
%      \begin{tikzpicture}[line width = \pgfdefaultlinewidth,
                    x = 0.625cm,
                    y = 0.625cm]

    \pgfmathsetmacro{\distance}{10}

    \node at (3,0) {
    $\bm{C} = (     \pause
        \bm{c}_1,   \pause
        \bm{c}_2,   \pause
        \bm{c}_3)$
    };

    \uncover<2->{
        \fill[fill opacity=0.5] (2,    2) circle (1.6);
        \node at (2,    2) [text shadow={[align=center,text width=3cm] at (2,    2) {$\bm{c}_1$}}] {$\bm{c}_1$};
    }
    \uncover<3->{
        \fill[fill opacity=0.5] (4,    2) circle (1.6);
        \node at (4,    2) [text shadow={[align=center,text width=3cm] at (4,    2) {$\bm{c}_2$}}] {$\bm{c}_2$};
    }
    \uncover<4->{
        \fill[fill opacity=0.5] (3,3.732) circle (1.6);
        \node at (3,3.732) [text shadow={[align=center,text width=3cm] at (3,3.732) {$\bm{c}_3$}}] {$\bm{c}_3$};
    }


    \uncover<5->{
    \path[draw=black!50,solid,line width=9mm,fill=black!50, preaction={-triangle 60,line width = 4mm,draw = black!50,shorten >=-10mm}] (3 + 3,2.577) -- (3 - 4.5 + \distance,2.577);
    \node[text shadow={[align=center,text width=3cm] at (2.7    + 0.5*\distance,2.577) {Decouple}}] at (2.7    + 0.5*\distance,2.577) {Decouple};

    \fill[fill = black!51] (2 + \distance,    2) circle (1.6);
    \fill[fill = black!51] (4 + \distance,    2) circle (1.6);
    \fill[fill = black!51] (3 + \distance,3.732) circle (1.6);

    \draw[white, line width = 1pt] (2 + \distance,    2) circle (1.6);
    \draw[white, line width = 1pt] (4 + \distance,    2) circle (1.6);
    \draw[white, line width = 1pt] (3 + \distance,3.732) circle (1.6);

    \node at (1.52 + \distance,1.72 ) [text shadow={[align=center,text width=3cm] at (1.52 + \distance,1.72 ) {$\bm{c}_1'$}}] {$\bm{c}_1'$};
    \node at (4.48 + \distance,1.72 ) [text shadow={[align=center,text width=3cm] at (4.48 + \distance,1.72 ) {$\bm{c}_2'$}}] {$\bm{c}_2'$};
    \node at (3    + \distance,4.29 ) [text shadow={[align=center,text width=3cm] at (3    + \distance,4.29 ) {$\bm{c}_3'$}}] {$\bm{c}_3'$};
    \node at (3    + \distance,2.577) [text shadow={[align=center,text width=3cm] at (3    + \distance,2.577) {$\bm{c}_4'$}}] {$\bm{c}_4'$};
    \node at (3    + \distance,1.447) [text shadow={[align=center,text width=3cm] at (3    + \distance,1.447) {$\bm{c}_5'$}}] {$\bm{c}_5'$};
    \node at (2.21 + \distance,3.03 ) [text shadow={[align=center,text width=3cm] at (2.21 + \distance,3.03 ) {$\bm{c}_6'$}}] {$\bm{c}_6'$};
    \node at (3.79 + \distance,3.03 ) [text shadow={[align=center,text width=3cm] at (3.79 + \distance,3.03 ) {$\bm{c}_7'$}}] {$\bm{c}_7'$};

    \node at (3 + \distance,0) {$\bm{C}' = (\bm{c}'_1,\bm{c}'_2,\cdots,\bm{c}'_7)$};
    }

    \onslide<1->
\end{tikzpicture} 
%    \end{center}
%\end{frame}

\begin{frame}{Decouple of Incident Consequences -- Step 3}
     For each $\bm{c}'_j \in \bm{C}'$, generate a corresponding auxiliary node $x_j$. According to the \textbf{traceability} of $\bm{C}'$
     \[
        \forall \bm{c}' \in \bm{C}', \exists \bm{c} \in \bm{C}, \bm{c}' \subseteq \bm{c}\text{,}
     \]
     \vspace{-15pt}\\
     there must be a consequence set $\bm{c}_i \in \bm{C}$ , where $\bm{c}'_j \subseteq \bm{c}_i$. \pause So, for each $\bm{c}'_j \in \bm{C}'$, we can find the incident set
     \[
        \bm{e}_j = (e_{i_1},e_{i_2},\cdots,e_{i_n})\text{.}
     \]
     \vspace{-15pt}\\\pause
     For each incident $e_k$ of the incident set $\bm{e}_j$, the corresponding consequence set $\bm{c}_k$ satisfies the following condition:
     \[
        \bm{c}'_j \subseteq \bm{c}_k\text{.}
     \]
     \vspace{-15pt}\\\pause
     Therefore, the parent nodes of the auxiliary node $x_j$ are incident nodes $e_{i_1},e_{i_2},\cdots,e_{i_n}$.
\end{frame}

\begin{frame}{Decouple of Incident Consequences -- Step 4}
    For each auxiliary node $x_j$, generate a conditional probability table. A typical conditional probability table of auxiliary node $x_j$ is shown as following table.\\[-15pt]
    \extrarowsep = 0.7mm
    \begin{center}
      \begin{tabu}to \textwidth{@{}X[c,2]@{}|*7{X[c]}@{}}
        $H(e_{i_1})$        & T & T & T & $\cdots$ & F & F & F\\
        $H(e_{i_2})$        & T & T & T & $\cdots$ & F & F & F\\
        $H(e_{i_3})$        & T & T & T & $\cdots$ & F & F & F\\
        $\vdots$            & $\vdots$ & $\vdots$ & $\vdots$ & $\ddots$ & $\vdots$ & $\vdots$ & $\vdots$\\
        $H(e_{i_{n-2}})$    & T & T & T & $\cdots$ & F & F & F\\
        $H(e_{i_{n-1}})$    & T & T & F & $\cdots$ & T & F & F\\
        $H(e_{i_n})$        & T & F & F & $\cdots$ & F & T & F\\
        \hline
        $H(x_j)$            & $1$ & $1$ & $1$ & $\cdots$ & $1$ & $1$ & $0$ \\
        $\overline{H}(x_j)$ & $0$ & $0$ & $0$ & $\cdots$ & $0$ & $0$ & $1$
      \end{tabu}
    \end{center}
\end{frame}

\subsection{Classification of Incident Consequences}
\begin{frame}{Classification of Incident Consequences}
    In this paper, there are three main kinds of incident consequences to be considered:
    \begin{itemize}
      \item \textbf{Harm to Humans:}\\[-5pt]
      \begin{itemize}
        \item[-] temporary harm,
        \item[-] permanent disability,
        \item[-] fatality.
      \end{itemize}
      \item \textbf{Environmental Pollution:}\\[-5pt]
      \begin{itemize}
        \item[-] air pollution,
        \item[-] soil contamination,
        \item[-] water pollution.
      \end{itemize}
      \item \textbf{Property Loss:}\\[-5pt]
      \begin{itemize}
        \item[-] damage of materials,
        \item[-] damage of products,
        \item[-] damage of equipment.
      \end{itemize}
    \end{itemize}
\end{frame}

\subsection{Quantification of Incident Consequences}
\begin{frame}{Quantification of Incident Consequences}
    \begin{itemize}
      \item \textbf{Harm to Humans $Q_H$:}\\
      If the decision-maker would like to increase the cost of an investment by $\Delta c$ to reduce the  probability of a fatality by $\Delta p$,
      \[
        Q_H = \Delta c/\Delta p\text{.}
      \]
      \item \textbf{Environmental Pollution $Q_E$:}\\
      The monetary loss of environmental pollution is defined as
      \[
        Q_E = {\it Penalty} + {\it Compensation} + {\it HarnessCost}\text{.}
      \]
      \item \textbf{Property Loss $Q_P$:}\\
      The cost of replacement is used to quantify the loss of property $Q_P$, such as the loss of  materials, products, and equipment.
    \end{itemize}
\end{frame}

\subsection{Calculation of Dynamic Risk}
\begin{frame}{Calculation of Dynamic Risk}
    Due to the following two reasons:
    \begin{itemize}
      \item there is no overlapping between the consequences of any two auxiliary nodes $x_i$ and $x_j$, $i\neq j$,
      \item the auxiliary nodes contain all the consequences of incidents,
    \end{itemize}
    
    the dynamic cybersecurity risk can be defined as 
    \[
        \risk = \sum_{i=1}^{m'}p(x_i)q(x_i)\text{,}
    \]
    
    where
    \begin{itemize}
      \item $p(x_i)$ is the occurrence probability of the auxiliary node $x_i$,
      \item $q(x_i)$ is the monetary loss of the auxiliary node $x_i$.
    \end{itemize}
     
\end{frame} 