\section{Hazardous Incident Prediction}
\subsection{The Bayesian Network Based Knowledge Modeling}
\begin{frame}{An Example of Multi-Step Attacks}

\end{frame}

\begin{frame}{Attack Level}
    In this paper, the Bayesian network is used to model the relationship between attacks and resources.
    \begin{center}
      \begin{tikzpicture}[line width = \pgfdefaultlinewidth,
                    attack/.style = {draw = red, fill = red, text = white, circle, font = \scriptsize, minimum size = 0.6cm, inner sep = 0pt},
                    resource/.style = {draw = green, fill = green, text = white, circle, font = \scriptsize, minimum size = 0.6cm, inner sep = 0pt}]

    \pgfmathsetmacro{\interval}{2}

    
    \node[attack] (a1) at (1, 0*\interval) {$\bm{a_1}$};
    \node[attack] (a2) at (3, 0*\interval) {$\bm{a_2}$};
    \pause

    \node[resource] (r1) at (1, 1*\interval) {$\bm{r_1}$};
    \node[resource] (r2) at (2, 1*\interval) {$\bm{r_2}$};
    \node[resource] (r3) at (3, 1*\interval) {$\bm{r_3}$};

    \foreach \i/\j in{1/1,
                      1/2,
                      2/1,
                      2/2,
                      2/3}
    {
        \draw[->, red] (a\i) -- (r\j);
    }
    \pause


    \node[attack] (a3) at (1, 2*\interval) {$\bm{a_3}$};
    \node[attack] (a4) at (3, 2*\interval) {$\bm{a_4}$};

    \foreach \i/\j in {1/3,
                       1/4,
                       2/3,
                       3/4}
    {
        \draw[->, green] (r\i) -- (a\j);
    }
    \pause

    \extrarowsep =1mm
    \node[anchor = north west] (CPTR2) at (4,2) {\scriptsize
        \begin{tabu}to 7cm{X[l,-1]@{}|@{}*4{X[c,-1]@{}}}

            $L(a_1)$ & T & T & F & F\\
            $L(a_2)$ & T & F & T & F \\
            \hline
            $O(r_2)$ & $o_{r_2,1}$ & $o_{r_2,2}$ & $o_{r_2,3}$ & $o_{r_2,4}$\\
            $\overline{O}(r_2)$ & $1-o_{r_2,1}$ & $1-o_{r_2,2}$ & $1-o_{r_2,3}$ & $1-o_{r_2,4}$
        \end{tabu}
    };
    \draw[->, dotted] (r2) to[out=-45, in=180] (CPTR2);
    \pause

    \node[anchor = north west] (CPTA4) at (4,4.5) {\scriptsize
        \begin{tabu}to 7cm{X[l,-1]@{}|@{}*4{X[c,-1]@{}}}

            $O(r_1)$ & T & T & F & F\\
            $O(r_3)$ & T & F & T & F\\
            \hline
            $L(a_4)$ & $\ell_{a_4,1}$ & $\ell_{a_4,2}$ & $\ell_{a_4,3}$ & $\ell_{a_4,4}$\\
            $\overline{L}(a_4)$ & $1-\ell_{a_4,1}$ & $1-\ell_{a_4,2}$ & $1-\ell_{a_4,3}$ & $1-\ell_{a_4,4}$
        \end{tabu}
    };
    \draw[->, dotted] (a4) to[out=-45, in=180] (CPTA4);


    \onslide<1->
\end{tikzpicture} 
    \end{center}

\end{frame}

\begin{frame}{Function Level}
    Function Tree Analysis is widely used to analyze the stability of control system, a typical function tree is shown in following figure.
    \begin{center}
      \begin{tikzpicture}[line width = \pgfdefaultlinewidth,
                    circuit logic US,
                    function/.style = {draw = blue, fill = blue, text = white, circle, font = \scriptsize, minimum size = 0.6cm, inner sep = 0pt}]
    \pause
    \node[function] (f1)  at (3.5,4) {$\bm{f_{1}}$};
    \pause
    \pause

    \node[and gate,point up,logic gate inputs=1234,logic gate inverted radius=1pt] (gate1) at (3.5,3) {};
    \draw (gate1.output) -- (f1);

    \node[function] (f2)  at (1.5,2) {$\bm{f_2}$};
    \node[function] (f3) at (1.5+4/3,2) {$\bm{f_{3}}$};
    \node[function] (f4) at (1.5+8/3,2) {$\bm{f_{4}}$};
    \node[function] (f5) at (5.5,2) {$\bm{f_{5}}$};

    \draw (f2.north) -- ++ (up:2mm) -| (gate1.input 1);
    \draw (f3.north) -- ++ (up:1mm) -| (gate1.input 2);
    \draw (f4.north) -- ++ (up:1mm) -| (gate1.input 3);
    \draw (f5.north) -- ++ (up:2mm) -| (gate1.input 4);

    \pause
    \pause

    \node[and gate,point up,logic gate inputs=nnin, logic gate inverted radius=1pt] (gate2) at (1.5,1) {};
    \draw (gate2.output) -- (f2);

    \node[function] (f6) at (0,0) {$\bm{f_6}$};
    \node[function] (f7) at (1,0) {$\bm{f_7}$};
    \node[function] (f8) at (2,0) {$\bm{f_8}$};
    \node[function] (f9) at (3,0) {$\bm{f_9}$};

    \draw (f6.north) -- ++ (up:2mm) -| (gate2.input 1);
    \draw (f7.north) -- ++ (up:1mm) -| (gate2.input 2);
    \draw (f8.north) -- ++ (up:1mm) -| (gate2.input 3);
    \draw (f9.north) -- ++ (up:2mm) -| (gate2.input 4);

    \pause
    \pause

    \node[or gate,point up,logic gate inputs=1234,logic gate inverted radius=1pt] (gate3) at (5.5,1) {};
    \draw (gate3.output) -- (f5);

    \node[function] (f10) at (4,0) {$\bm{f_{10}}$};
    \node[function] (f11) at (5,0) {$\bm{f_{11}}$};
    \node[function] (f12) at (6,0) {$\bm{f_{12}}$};
    \node[function] (f13) at (7,0) {$\bm{f_{13}}$};

    \draw (f10.north) -- ++ (up:2mm) -| (gate3.input 1);
    \draw (f11.north) -- ++ (up:1mm) -| (gate3.input 2);
    \draw (f12.north) -- ++ (up:1mm) -| (gate3.input 3);
    \draw (f13.north) -- ++ (up:2mm) -| (gate3.input 4);

    \onslide<1->
\end{tikzpicture} 
    \end{center}
    
    \begin{overlayarea}{\textwidth}{1cm}
    \only<3-4>{
        \[
            F_1 = F_2F_3F_4F_5
        \]
    }
    \only<5-6>{
        \[
            F_2 = F_6F_7\overline{F_8}F_9
        \]
    }
    \only<7-8>{
        \[
            F_5 = F_{10}+F_{11}+F_{12}+F_{13}
        \]
    }    
    \end{overlayarea}
\end{frame}

\begin{frame}{Function Level}
    To simplify the inference, the function tree is converted into Bayesian network, which is shown in following figure.\\
    \begin{overlayarea}{\textwidth}{6cm}
    \begin{center}
      \only<1>{
        \begin{tikzpicture}[line width = \pgfdefaultlinewidth,
                    circuit logic US,
                    function/.style = {draw = blue, fill = blue, text = white, circle, font = \scriptsize, minimum size = 0.6cm, inner sep = 0pt}]

    \node[function] (f1) at (2.5,4) {$\bm{f_{1}}$};

    \node[and gate,point up,logic gate inputs=1234,logic gate inverted radius=1pt] (gate1) at (2.5,3) {};
    \draw (gate1.output) -- (f1);

    \node[function] (f2) at (1,2) {$\bm{f_2}$};
    \node[function] (f3) at (2,2) {$\bm{f_{3}}$};
    \node[function] (f4) at (3,2) {$\bm{f_{4}}$};
    \node[function] (f5) at (4,2) {$\bm{f_{5}}$};

    \draw (f2.north) -- ++ (up:2mm) -| (gate1.input 1);
    \draw (f3.north) -- ++ (up:1mm) -| (gate1.input 2);
    \draw (f4.north) -- ++ (up:1mm) -| (gate1.input 3);
    \draw (f5.north) -- ++ (up:2mm) -| (gate1.input 4);


    \node[function] (f1') at (9.5,4) {$\bm{f_{1}}$};

    \node[function] (f2') at (8,2) {$\bm{f_2}$};
    \node[function] (f3') at (9,2) {$\bm{f_3}$};
    \node[function] (f4') at (10,2) {$\bm{f_4}$};
    \node[function] (f5') at (11,2) {$\bm{f_5}$};

    \path[draw=black!50,solid,line width=9mm,fill=black!50, preaction={-triangle 60,line width = 4mm,draw = black!50,shorten >=-10mm}] (4.3,3) -- (7,3);

    \draw[->] (f2') -- (f1');
    \draw[->] (f3') -- (f1');
    \draw[->] (f4') -- (f1');
    \draw[->] (f5') -- (f1');

    \extrarowsep =1mm
    \node[anchor = north] (CPT) at (6,1.5) {\scriptsize
        \begin{tabu}{X[4,c]|*{16}{X[c]}}
          $F(f_2)$ & T & T & T & T & T & T & T & T & F & F & F & F & F & F & F & F \\
          $F(f_3)$ & T & T & T & T & F & F & F & F & T & T & T & T & F & F & F & F \\
          $F(f_4)$ & T & T & F & F & T & T & F & F & T & T & F & F & T & T & F & F \\
          $F(f_5)$ & T & F & T & F & T & F & T & F & T & F & T & F & T & F & T & F \\
          \hline
          $F(f_1)$ & $1$ & $0$ & $0$ & $0$ & $0$ & $0$ & $0$ & $0$ & $0$ & $0$ & $0$ & $0$ & $0$ & $0$ & $0$ & $0$ \\
          $\overline{F}(f_1)$ & $0$ & $1$ & $1$ & $1$ & $1$ & $1$ & $1$ & $1$ & $1$ & $1$ & $1$ & $1$ & $1$ & $1$ & $1$ & $1$
        \end{tabu}
    };

    \onslide<1->
\end{tikzpicture} 
      }
      \only<2>{
        \begin{tikzpicture}[line width = \pgfdefaultlinewidth,
                    circuit logic US,
                    function/.style = {draw = blue, fill = blue, text = white, circle, font = \scriptsize, minimum size = 0.6cm, inner sep = 0pt}]

    \node[function] (f2) at (2.5,4) {$\bm{f_{2}}$};

    \node[and gate,point up,logic gate inputs=nnin,logic gate inverted radius=1pt] (gate1) at (2.5,3) {};
    \draw (gate1.output) -- (f1);

    \node[function] (f6) at (1,2) {$\bm{f_6}$};
    \node[function] (f7) at (2,2) {$\bm{f_{7}}$};
    \node[function] (f8) at (3,2) {$\bm{f_{8}}$};
    \node[function] (f9) at (4,2) {$\bm{f_{9}}$};

    \draw (f6.north) -- ++ (up:2mm) -| (gate1.input 1);
    \draw (f7.north) -- ++ (up:1mm) -| (gate1.input 2);
    \draw (f8.north) -- ++ (up:1mm) -| (gate1.input 3);
    \draw (f9.north) -- ++ (up:2mm) -| (gate1.input 4);


    \node[function] (f2') at (9.5,4) {$\bm{f_{2}}$};

    \node[function] (f6') at (8,2) {$\bm{f_6}$};
    \node[function] (f7') at (9,2) {$\bm{f_7}$};
    \node[function] (f8') at (10,2) {$\bm{f_8}$};
    \node[function] (f9') at (11,2) {$\bm{f_9}$};

    \path[draw=black!50,solid,line width=9mm,fill=black!50, preaction={-triangle 60,line width = 4mm,draw = black!50,shorten >=-10mm}] (4.3,3) -- (7,3);

    \draw[->] (f6') -- (f2');
    \draw[->] (f7') -- (f2');
    \draw[->] (f8') -- (f2');
    \draw[->] (f9') -- (f2');

    \extrarowsep =1mm
    \node[anchor = north] (CPT) at (6,1.5) {\scriptsize
        \begin{tabu}{X[4,c]|*{16}{X[c]}}
          $F(f_6)$ & T & T & T & T & T & T & T & T & F & F & F & F & F & F & F & F \\
          $F(f_7)$ & T & T & T & T & F & F & F & F & T & T & T & T & F & F & F & F \\
          $F(f_8)$ & T & T & F & F & T & T & F & F & T & T & F & F & T & T & F & F \\
          $F(f_9)$ & T & F & T & F & T & F & T & F & T & F & T & F & T & F & T & F \\
          \hline
          $F(f_2)$ & $0$ & $0$ & $1$ & $0$ & $0$ & $0$ & $0$ & $0$ & $0$ & $0$ & $0$ & $0$ & $0$ & $0$ & $0$ & $0$ \\
          $\overline{F}(f_2)$ & $1$ & $1$ & $0$ & $1$ & $1$ & $1$ & $1$ & $1$ & $1$ & $1$ & $1$ & $1$ & $1$ & $1$ & $1$ & $1$
        \end{tabu}
    };

    \onslide<1->
\end{tikzpicture} 
      }
      \only<3>{
        \input{Figures/Hazardous.Incident.Prediction/FT.to.BN.Gate3.tex}
      }
    \end{center}
    \end{overlayarea}
\end{frame}

\begin{frame}{Incident Level}
\end{frame}

\subsection{Incident Prediction}
\begin{frame}{Collection of Evidence}
\end{frame}

\begin{frame}{Calculation of Incident Probability}
\end{frame} 