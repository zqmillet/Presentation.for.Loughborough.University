\section{Conclusion and Prospect}
\subsection{Conclusion}
\begin{frame}{Conclusion}
\begin{itemize}[<+->]
  \item By considering the characteristics of ICSs, a novel multi-level Bayesian network was proposed, which integrated a knowledge of attack, system function, and hazardous incident.
  \item The attack knowledge and system knowledge were combined to analyze the potential impact of attacks, so the proposed approach had the ability of assessing the risk caused by unknown attacks.
  \item A unified quantification approach for a variety of consequences of industrial accidents was introduced. Furthermore, the proposed approach could eliminate the error of risk caused by the overlapping amongst hazardous incidents.
  \item By using a simplified chemical reactor control system in Matlab environment, the designed dynamic risk assessment approach was verified.
\end{itemize}
\end{frame}

\subsection{Prospect}
\begin{frame}{Prospect}
    There are some shortcomings of the proposed risk assessment approach need to be improved.
    \begin{itemize}[<+->]
      \item \textbf{Current research work has no ability for self-learning.} 
      \item \textbf{The sub-second computation time cannot meet some hard real-time systems requirements.}
    \end{itemize}
    \uncover<+->{In the future, a dynamic cybersecurity risk assessment, which can automatically adjust the conditional probability and structure of the multi-level Bayesian network by analyzing the real-time data, will be researched, and several approximate inference methods will be attempted in the risk assessment.}
\end{frame}

