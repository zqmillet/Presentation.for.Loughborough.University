\documentclass{article}
\usepackage[noxcolor]{beamerarticle}
\usepackage{pgfpages}
\usepackage{enumitem}
\usepackage{geometry}
\usepackage{fontspec}
\usepackage{mathrsfs}
\usepackage{amsmath}
\usepackage{bm}
\usepackage{mathtools}
\usepackage{wrapfig}
\usepackage{calc}
\usepackage{adjustbox}
\usepackage{xifthen}

\setjobnamebeamerversion{maintrans}

\setmainfont{Times New Roman}

\setlength\parindent{1em}
\linespread{1.2}

\setenumerate[1]{itemsep=0pt,partopsep=0pt,parsep=0pt,topsep=0pt,leftmargin=1.1em}
\setitemize[1]{itemsep=0pt,partopsep=0pt,parsep=0pt,topsep=0pt,leftmargin=1em}
\setdescription{itemsep=0pt,partopsep=0pt,parsep=0pt,topsep=0pt,leftmargin=1em}

\newlength\qiqiheight
\setlength{\intextsep}{0pt}%
\setlength{\columnsep}{11pt}%
\setlength{\parskip}{10pt}

\newcommand{\risk}{\mathscr{R}}

\newcommand{\addnote}[2]{
    \begin{adjustbox}{minipage=0.5\textwidth, gstore totalheight=\qiqiheight, gobble}
        \setlength{\parskip}{0pt}
        \setlength\parindent{1em}
        \vspace{0pt}
        #2
    \end{adjustbox}
    \ifthenelse{\lengthtest{\qiqiheight > 150pt}}
    {
        \begin{wrapfigure}{l}{0.45\textwidth}
            \vspace{0pt}
            \fbox{\includeslide[width = 0.45\textwidth]{#1}}
        \end{wrapfigure}\setlength{\parskip}{0pt}
        #2
    }
    {
        \noindent\begin{minipage}[t]{0.45\textwidth}
            \vspace{0pt}
            \fbox{\includeslide[width = \textwidth]{#1}}
        \end{minipage}\hspace{10pt}
        \begin{minipage}[t]{0.5\textwidth}
            \setlength{\parskip}{0pt}
            \setlength\parindent{1em}
            \vspace{0pt}
            #2
        \end{minipage}
    }\rule{0pt}{0pt}\\\vfill
}

\geometry{left=3.5cm,right=3.5cm, top= 3.5cm}
\begin{document}

\addnote{Title}{
    Hello everyone, my name is Zhang Qi, and I am the Ph.D student of Professor Zhou Chunjie. I am very glad to be invited by Professor Yang Shuanghuang to make a presentation about my recent research.


    My research interests are related to risk assessment and decision-making for industrial control systems.     The title of my presentation is ``Multi-Model Based Incident Prediction and Risk Assessment in Dynamic Cybersecurity Protection for Industrial Control Systems''.
}

\addnote{Outlines}{
    My presentation is separated into six parts:
    \begin{itemize}
      \item Firstly, I will introduce the background and the problems of risk assessment for industrial control systems.
      \item Secondly, I will give the architecture of our risk assessment solution for industrial control systems.
      \item Thirdly, I will elaborate the detail of our method.
      \item Then, I will show you the effectiveness of our approach by using a numerical simulation.
      \item At last, I will discuss the problems of our approach and introduce the future works.
    \end{itemize}
}

\addnote{Section: Introduction}{
    In this part, I will introduce the development history and the cybersecurity issues of industrial control systems. And, I will compare the cybersecurity issues of industrial control systems and traditional IT systems.
}

\addnote{Introduction: Development of ICSs}{
    There are four great changes in the development of industrial control systems:
    \begin{itemize}
      \item Machine Age
      \item Semi-automatic Age
      \item Automatic Age
      \item Intelligent Age
    \end{itemize}

    From this figure, we can see that with the development of industrial control systems, the degree of automation is increasing. Intelligence and networking are the development trend of industrial control systems.
}

\addnote{Introduction: ICSs are Important}{
    Nowadays, the industrial control systems have been widely applied in various industry, and they are becoming more and more important for the national economy and our life.

    As mentioned before, the industrial control systems are evolving towards intelligence and networking. The rapid development of the industrial control systems reduce the difficulty of the development and the cost of construction, on the other hand, it has also introduced the cybersecurity issues into the industrial control systems.
}

\addnote{Introduction: ICSs are under Attacks}{
    For example, in 2010, the Stuxnet attacked Iran's nuclear power plants and ruined almost one-fifth of Iran's nuclear centrifuges. In 2013, Israel Haifa highway control system  was attacked by hackers, which caused massive traffic congestion in the city which lead great loss and serious subsequent problems.

    According to the statistical data from ``Year in Review 2014'' published by the ICS-CERT which is short for ``Industrial Control Systems Cyber Emergency Response Team'', the number of attacks for industrial control systems increases year by year. In 2010, there were only 39 security incidents of industrial control systems, but in 2014, this number has grown to 245.

    Unlike traditional IT systems, the security incidents of industrial control systems can cause irreparable harm to the physical systems being controlled and to the people dependent on them. Basically, protecting industrial control systems against cyber-attacks is vital to both the economy and stability of a nation. Therefore, the cybersecurity issue of industrial control systems must be taken seriously and solved as soon as possible.
}

\addnote{Introduction: Problem of Timeliness and Availability}{
    In recent years, considerable researches have been undertaken to study cybersecurity risk assessment methods. However, the cybersecurity risk assessment in the IT domain is not entirely applicable to industrial control systems because industrial control systems are relatively different from traditional IT systems in some aspects.

    Firstly, the cybersecurity objectives are different. Traditional IT systems first require an ensuring of confidentiality, then integrity, and finally availability. For industrial control systems, in contrast, the priorities of these three security objectives are first availability, then integrity, and finally confidentiality, because the timeliness and availability are the primary concerns. The malicious attacks induce the cybersecurity risk to industrial control systems by demolishing the timeliness and availability. Therefore, the risk assessment of industrial control systems needs a novel risk propagation analysis approach.
}

\addnote{Introduction: Problem of Overlapping amongst Consequences}{
    The majority of existing quantitative risk assessment approaches used this definition to calculate the risk, where $S(e_i)$ is the severity of the incident $e_i$ and $P(e_i)$ is the probability of the incident $e_i$.

     It is also worth noting that there is a problem when this definition is used in industrial control systems risk assessment. This is due to the fact that, for industrial control systems, different hazardous incidents may cause the same consequence; whereby, using this definition to assess risk will cause the severity of the same consequence to be accumulated multiple times. As a result, there is an error in the risk assessment, which cannot be ignored. Even worse, the decision-making may generate a wrong policy with this inaccurate risk value.

     For example, incident $e_1$ is the temperature anomaly of reactor, incident $e_2$ is the explosion of reactor, when $e_1$ or $e_2$ happens, the product will be damaged. Assume that $P(e_1) = 1$, so $P(e_2) = p_1$, then
     \[
        \risk = S(e_1) + p_1S(e_2) = S(e_1) + p_1S(e_1) = (1+p_1)S(e_1) \geq S(e_1)\text{.}
     \]

     It is obviously wrong, because the risk of system can't be larger than the total value of all assets.
}

\addnote{Introduction: Problem of Unknown Attacks}{
    As continuous operation systems, the industrial control systems cannot tolerate frequent software patching or updates. This causes the database of attack signatures to lag far behind the rapid development of attacks. With this defect, several intrusion detection system based misuse detections would miss the unknown attacks.

    On the other hand, without the information about unknown attacks, such as purposes, consequences, and further steps, these unknown attacks and their consequences cannot be predicted accurately. As a result, the risk assessment module will generate erroneous risk value, which may lead to a wrong decision.
}

\addnote{Section: Architecture}{
    Based on the above analysis, the requirements of cybersecurity risk assessment for industrial control systems can be summarized. The risk assessment of industrial control systems needs:
    \begin{itemize}
      \item a novel and targeted risk model to analyze the risk propagation,
      \item a unified quantification approach to calculate the risk quantitatively without the error caused by overlapping amongst consequences,
      \item the ability of assessing the risk caused by unknown attacks without the corresponding attack knowledge.
    \end{itemize}
}

\addnote{Architecture: Cybersecurity Risk Assessment for ICSs}{
    To meet the requirement of the risk assessment for industrial control systems, a dynamic cybersecurity risk assessment based on the multi-model is proposed.

    To analyze the propagation of cybersecurity risk, the attack model, the function model, and the incident model are considered. Then, these three models are converted into a multi-level Bayesian network. This Bayesian network has three levels: the attack level, the function level, and the incident level.

    There are two kinds of inputs for the dynamic cybersecurity risk assessment: attack evidence and anomaly evidence. Attack evidence, which contains information about the type, target, and timestamp of the detected attack, is derived from intrusion detection system. Anomaly evidence, containing the information of the anomaly, such as the invalidation of a function, the occurrence of a hazardous incident, etc., can be obtained from the supervisor system of industrial control systems.

    The dynamic cybersecurity risk assessment is divided into two phases: the hazardous incident prediction and the risk assessment. During the hazardous incident prediction phase, attack evidence and anomaly evidence are collected and marked in the multi-level Bayesian network. Then, probabilities of all the potential hazardous incidents can be calculated by analyzing the collected evidences and the multi-level Bayesian network. During the risk assessment phase, the consequences of the hazardous incidents are first classified, and then each type of consequence is quantified in the same unit. Secondly, the overlapping amongst hazardous incidents must be addressed, so the error caused by multiple accumulation of consequences can be eliminated. Finally, the probabilities and consequences of the hazardous incidents are combined into the cybersecurity risk.
}

\addnote{Section: Hazardous Incident Prediction}{
    Next, I will elaborate the proposed approach of risk assessment for industrial control systems from two parts:
    \begin{itemize}
      \item hazardous incident prediction
      \item dynamic risk assessment
    \end{itemize}
}

\addnote{Hazardous Incident Prediction: Attack Level}{
    In the proposed approach, the Bayesian network is used to model the relationship between attacks and resources.

    The left figure shows a typical Bayesian network of multi-step attack. In this Bayesian network, the attack nodes, which are colored red, represent attack strategies. the resource nodes, which are color green, represent resources. The enforcement of an attack strategy need some conditions. Only the conditions of an attack strategy is satisfied, may this attack strategy be launched. One the other hands, the enforcement of an attack strategy may obtain another resources. So, using these two kinds of nodes, the Bayesian network can model the multi-step attack.

    The Bayesian network uses the conditional probability table to describe the reachable probability. For example, attack node $a_4$ has two conditions $r_1$ and $r_3$. The first column of the conditional probability table of node $a_4$ shows that when the attacker obtain the resources $r_1$ and $r_3$, the probability that he launches attack $a_4$ is $\ell_{a_4,1}$. Similarly, if he only has resource $r_1$, the probability is $\ell_{a_4,2}$.
}

\addnote{Hazardous Incident Prediction: Function Tree F1}{
    Function Tree Analysis is widely used to analyze the stability of control system, a typical function tree is shown in following figure.

    If the relationship amongst the functions lowercase $f_1$, $f_2$, $f_3$, $f_4$ and $f_5$ is uppercase $F_1$ equals $F_2F_3F_4F_5$. In this slide, there are two kinds of letter {\bf F}, where the lowercase $f$ represents the system function, the uppercase $F$ represents the status of system function $f$. For example, the uppercase $F_1$ equals \texttt{True} means that the corresponding system function lowercase $f_1$ is running normally, the uppercase $F_1$ equals \texttt{False} means that there is something wrong with the corresponding system function lowercase $f_1$.

    Let's go back to the relationship amongst the functions lowercase $f_1$, $f_2$, $f_3$, $f_4$ and $f_5$, if the relationship amongst the functions lowercase $f_1$, $f_2$, $f_3$, $f_4$ and $f_5$ is uppercase $F_1$ equals $F_2F_3F_4F_5$. The function tree uses an and-gate to describe this relationship.
}

\addnote{Hazardous Incident Prediction: Function Tree F2}{
    If the relationship amongst the functions lowercase $f_2$, $f_6$, $f_7$, $f_8$ and $f_9$ is uppercase $F_2$ equals $F_6F_7\overline{F}_8F_9$. The function tree will uses an appropriate logical gate to describe this kind of relationship.
}

\addnote{Hazardous Incident Prediction: Function Tree F5}{
    Similarly, if the relationship amongst the functions lowercase $f_5$, $f_{10}$, $f_{11}$, $f_{12}$ and $f_{13}$ is uppercase $F_5$ equals $F_{10} + F_{11} + F_{12} + F_{13}$. The function tree will uses an or-gate to describe this kind of relationship.
}

\addnote{Hazardous Incident Prediction: Function Tree to Bayesian Network Gate 1}{
    To simplify the inference, the function tree is converted into the Bayesian network, which is shown in following figure.

    This and gate can be converted to a Bayesian network, in which $f_2$, $f_3$, $f_4$ and $f_5$ is the parent nodes of $f_1$. Of cause, a conditional probability table is needed, too.
}

\addnote{Hazardous Incident Prediction: Function Tree to Bayesian Network Gate 2}{
    This kinds of gate can be also converted into a Bayesian network, but the conditional probability table is different. In fact, all kinds of logical gates can be converted into corresponding Bayesian networks.
}

\addnote{Hazardous Incident Prediction: Function Tree to Bayesian Network Gate 3}{
    For example, the or-gate can be converted into the following Bayesian network.
}

\addnote{Hazardous Incident Prediction: Comparison of Fault Tree and Bayesian Network Question}{
    Now, let me digress for a moment. There is a question: can the Bayesan network be converted into the function tree?

    The answer is YES, but not all the Bayesian networks can be converted into the corresponding function trees.

    For example, the following conditional probability table can't be converted into a function tree.
}

\addnote{Hazardous Incident Prediction: Comparison of Fault Tree and Bayesian Network Answer}{
    Because the conditional probability table of the Bayesian network contains more information than the logical gate of the fault tree. In other words, the logical gate cannot always accurately describe the relationship amongst functions.

    the following conditional probability is an example.
}

\addnote{Hazardous Incident Prediction: Incident Level}{
    The occurrence of one incident may cause another incidents, in the proposed approach, the Bayesian network is also used to model the causal relationship amongst the potential incidents. A typical Bayesian network of incident is shown in following figure.

    Like the attack level, the incident node also needs a conditional probability table to describe the relationship amongst it and its parent nodes.
}

\addnote{Hazardous Incident Prediction: Information Transfer from Attack to Function}{
    The attack level, the function level and the incident level have been introduced. Now let's talk about the information transfer between levels.

    The cyber attacks can lead to system function failures, and the function failures may cause the industrial incidents. To analyze the risk propagation, an information transfer is necessary amongst the three aforementioned layers.
    
    The following figures show the information transfer between attack level and function level.
}

\addnote{Hazardous Incident Prediction: Information Transfer from Function to Incident}{
    The following figures show the information transfer between function level and incident level.
}

\addnote{Hazardous Incident Prediction: Collection of Evidence}{

}

\addnote{Hazardous Incident Prediction: Calculation of Incident Probability}{

}

\addnote{Section: Dynamic Risk Assessment}{

}

\addnote{Dynamic Risk Assessment: Decouple of Incident Consequences Step 1}{

}

\addnote{Dynamic Risk Assessment: Decouple of Incident Consequences Step 2}{

}

\addnote{Dynamic Risk Assessment: Decouple of Incident Consequences Step 3}{

}

\addnote{Dynamic Risk Assessment: Decouple of Incident Consequences Step 4}{

}

\addnote{Dynamic Risk Assessment: Classification of Incident Consequences}{

}

\addnote{Dynamic Risk Assessment: Quantification of Incident Consequences}{

}

\addnote{Dynamic Risk Assessment: Calculation of Dynamic Risk}{

}

\addnote{Section: Simulation}{

}

\addnote{Simulation: Control Structure of Chemical Reactor}{

}

\addnote{Simulation: Structure of Simulation Platform}{

}

\addnote{Simulation: Multi-Level Bayesian Network of Reactor}{

}

\addnote{Simulation: Evidences List}{

}

\addnote{Simulation: Quantification of Consequences}{

}

\addnote{Simulation: Curvers of Cybersecurity Risk and Incident Probability}{

}

\addnote{Simulation: Ability to Deal with the Unknown Attacks}{

}

\addnote{Simulation: Curve of Execution Time Distribution}{

}

\addnote{Simulation: Curves of Risk Assessment Algorithm Scalability}{

}

\addnote{Section: Conclusion and Prospect}{

}

\addnote{Conclusion and Prospect: Conclusion}{

}

\addnote{Conclusion and Prospect: Prospect}{

}

\addnote{Section: Thank You}{

}

\addnote{Thank You: Thank You}{

}

\addnote{Section: Questions}{

}



\end{document}
