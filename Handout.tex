\documentclass{article}
\usepackage[noxcolor]{beamerarticle}
\usepackage{pgfpages}
\usepackage{scrextend}
\deffootnote[1.8em]{0pt}{1.6em}{\makebox[1.8em][l]{\thefootnotemark.}}
\usepackage{enumitem}
\usepackage{geometry}
\usepackage{fontspec}
\usepackage{mathrsfs}
\usepackage{amsmath}
\usepackage{bm}
\usepackage{mathtools}
\usepackage{wrapfig}
\usepackage{calc}
\usepackage{adjustbox}
\usepackage{xifthen}

\setjobnamebeamerversion{maintrans}

\setmainfont{Times New Roman}
\setmonofont{Courier New}
\newfontfamily\pr{Arial}

\setlength\parindent{1em}
\linespread{1.2}

\setenumerate[1]{itemsep=0pt,partopsep=0pt,parsep=0pt,topsep=0pt,leftmargin=1.1em}
\setitemize[1]{itemsep=0pt,partopsep=0pt,parsep=0pt,topsep=0pt,leftmargin=1em}
\setdescription{itemsep=0pt,partopsep=0pt,parsep=0pt,topsep=0pt,leftmargin=1em}

\newlength\qiqiheight
\setlength{\intextsep}{0pt}%
\setlength{\columnsep}{11pt}%
\setlength{\parskip}{10pt}

\newcommand{\risk}{\mathscr{R}}

\newcommand{\addnote}[2]{
    \begin{adjustbox}{minipage=0.5\textwidth, gstore totalheight=\qiqiheight, gobble}
        \setlength{\parskip}{0pt}
        \setlength\parindent{1em}
        \vspace{0pt}
        #2
    \end{adjustbox}
    \ifthenelse{\lengthtest{\qiqiheight > 150pt}}
    {
        \begin{minipage}{\textwidth}\setlength\parindent{1em}
            \begin{wrapfigure}{l}{0.45\textwidth}
                \vspace{0pt}
                \fbox{\includeslide[width = 0.45\textwidth]{#1}}
            \end{wrapfigure}\setlength{\parskip}{0pt}
            #2
        \end{minipage}
    }
    {
        \begin{minipage}{\textwidth}\setlength\parindent{1em}
            \noindent\begin{minipage}[t]{0.45\textwidth}
                \vspace{0pt}
                \fbox{\includeslide[width = \textwidth]{#1}}
            \end{minipage}\hspace{10pt}
            \begin{minipage}[t]{0.5\textwidth}
                \setlength{\parskip}{0pt}
                \setlength\parindent{1em}
                \vspace{0pt}
                #2
            \end{minipage}
        \end{minipage}
    }\rule{0pt}{0pt}\\[10pt]
}

\newcommand{\pronunciation}[1]{
    \footnote{\pr #1}
}
\renewcommand{\thempfootnote}{\Roman{mpfootnote}}

\geometry{left=3.5cm,right=3.5cm, top= 3.5cm}
\begin{document}

\addnote{Title}{
    Hello everyone, my name is Zhang Qi, and I am the Ph.D student of Professor Zhou Chunjie. I am very glad to be invited by Professor Yang Shuanghuang to make a presentation about my recent research.


    My research interests are related to risk assessment and decision-making for industrial control systems.     The title of my presentation is ``Multi-Model Based Incident Prediction and Risk Assessment in Dynamic Cybersecurity Protection for Industrial Control Systems''.
}

\addnote{Outlines}{
    My presentation is separated into six parts:
    \begin{itemize}
      \item Firstly, I will introduce the background and the problems of risk assessment for industrial control systems.
      \item Secondly, I will give the architecture of our risk assessment solution for industrial control systems.
      \item Thirdly, I will elaborate the detail of our method.
      \item Then, I will show you the effectiveness of our approach by using a numerical simulation.
      \item At last, I will discuss the problems of our approach and introduce the future works.
    \end{itemize}
}

\addnote{Section: Introduction}{
    In this part, I will introduce the development history and the cybersecurity issues of industrial control systems. And, I will compare the cybersecurity issues of industrial control systems and traditional IT systems.
}

\addnote{Introduction: Development of ICSs}{
    There are four great changes in the development of industrial control systems:
    \begin{itemize}
      \item Machine Age
      \item Semi-automatic Age
      \item Automatic Age
      \item Intelligent Age
    \end{itemize}

    From this figure, we can see that with the development of industrial control systems, the degree of automation is increasing. Intelligence and networking are the development trend of industrial control systems.
}

\addnote{Introduction: ICSs are Important}{
    Nowadays, the industrial control systems have been widely applied in various industry, and they are becoming more and more important for the national economy and our life.

    As mentioned before, the industrial control systems are evolving towards intelligence and networking. The rapid development of the industrial control systems reduce the difficulty of the development and the cost of construction, on the other hand, it has also introduced the cybersecurity issues into the industrial control systems.
}

\addnote{Introduction: ICSs are under Attacks}{
    For example, in 2010, the Stuxnet attacked Iran's nuclear power plants and ruined almost one-fifth of Iran's nuclear centrifuges. In 2013, Israel Haifa highway control system  was attacked by hackers, which caused massive traffic congestion in the city which lead great loss and serious subsequent problems.

    According to the statistical data from ``Year in Review 2014'' published by the ICS-CERT which is short for ``Industrial Control Systems Cyber Emergency Response Team'', the number of attacks for industrial control systems increases year by year. In 2010, there were only 39 security incidents of industrial control systems, but in 2014, this number has grown to 245.

    Unlike traditional IT systems, the security incidents of industrial control systems can cause irreparable harm to the physical systems being controlled and to the people dependent on them. Basically, protecting industrial control systems against cyber-attacks is vital to both the economy and stability of a nation. Therefore, the cybersecurity issue of industrial control systems must be taken seriously and solved as soon as possible.
}

\addnote{Introduction: Problem of Timeliness and Availability}{
    In recent years, considerable researches have been undertaken to study cybersecurity risk assessment methods. However, the cybersecurity risk assessment in the IT domain is not entirely applicable to industrial control systems because industrial control systems are relatively different from traditional IT systems in some aspects.

    Firstly, the cybersecurity objectives are different. Traditional IT systems first require an ensuring of confidentiality, then integrity, and finally availability. For industrial control systems, in contrast, the priorities of these three security objectives are first availability, then integrity, and finally confidentiality, because the timeliness and availability are the primary concerns. The malicious attacks induce the cybersecurity risk to industrial control systems by demolishing the timeliness and availability. Therefore, the risk assessment of industrial control systems needs a novel risk propagation analysis approach.
}

\addnote{Introduction: Problem of Overlapping amongst Consequences}{
    The majority of existing quantitative risk assessment approaches used this definition to calculate the risk, where $S(e_i)$ is the severity of the incident $e_i$ and $P(e_i)$ is the probability of the incident $e_i$.

     It is also worth noting that there is a problem when this definition is used in industrial control systems risk assessment. This is due to the fact that, for industrial control systems, different hazardous incidents may cause the same consequence; whereby, using this definition to assess risk will cause the severity of the same consequence to be accumulated multiple times. As a result, there is an error in the risk assessment, which cannot be ignored. Even worse, the decision-making may generate a wrong policy with this inaccurate risk value.

     For example, incident $e_1$ is the temperature anomaly of reactor, incident $e_2$ is the explosion of reactor, when $e_1$ or $e_2$ happens, the product will be damaged. Assume that $P(e_1) = 1$, so $P(e_2) = p_1$, then
     \[
        \risk = S(e_1) + p_1S(e_2) = S(e_1) + p_1S(e_1) = (1+p_1)S(e_1) \geq S(e_1)\text{.}\pronunciation{the risk is equal to capital S e sub 1 plus p sub 1 multiplied by capital S e sub 2 is equal to capital S e sub 1 plus p sub 1 multiplied by capital S e sub 1 = 1 plus p sub 1 multiplied by capital S e sub 1}
     \]

     It is obviously wrong, because the risk of system can't be larger than the total value of all assets.
}

\addnote{Introduction: Problem of Unknown Attacks}{
    As continuous operation systems, the industrial control systems cannot tolerate frequent software patching or updates. This causes the database of attack signatures to lag far behind the rapid development of attacks. With this defect, several intrusion detection system based misuse detections would miss the unknown attacks.

    On the other hand, without the information about unknown attacks, such as purposes, consequences, and further steps, these unknown attacks and their consequences cannot be predicted accurately. As a result, the risk assessment module will generate erroneous risk value, which may lead to a wrong decision.
}

\addnote{Section: Architecture}{
    Based on the above analysis, the requirements of cybersecurity risk assessment for industrial control systems can be summarized. The risk assessment of industrial control systems needs:
    \begin{itemize}
      \item a novel and targeted risk model to analyze the risk propagation,
      \item a unified quantification approach to calculate the risk quantitatively without the error caused by overlapping amongst consequences,
      \item the ability of assessing the risk caused by unknown attacks without the corresponding attack knowledge.
    \end{itemize}
}

\addnote{Architecture: Cybersecurity Risk Assessment for ICSs}{
    To meet the requirement of the risk assessment for industrial control systems, a dynamic cybersecurity risk assessment based on the multi-model is proposed.

    To analyze the propagation of cybersecurity risk, the attack model, the function model, and the incident model are considered. Then, these three models are converted into a multi-level Bayesian network. This Bayesian network has three levels: the attack level, the function level, and the incident level.

    There are two kinds of inputs for the dynamic cybersecurity risk assessment: attack evidence and anomaly evidence. Attack evidence, which contains information about the type, target, and timestamp of the detected attack, is derived from intrusion detection system. Anomaly evidence, containing the information of the anomaly, such as the invalidation of a function, the occurrence of a hazardous incident, etc., can be obtained from the supervisor system of industrial control systems.

    The dynamic cybersecurity risk assessment is divided into two phases: the hazardous incident prediction and the risk assessment. During the hazardous incident prediction phase, attack evidence and anomaly evidence are collected and marked in the multi-level Bayesian network. Then, probabilities of all the potential hazardous incidents can be calculated by analyzing the collected evidences and the multi-level Bayesian network. During the risk assessment phase, the consequences of the hazardous incidents are first classified, and then each type of consequence is quantified in the same unit. Secondly, the overlapping amongst hazardous incidents must be addressed, so the error caused by multiple accumulation of consequences can be eliminated. Finally, the probabilities and consequences of the hazardous incidents are combined into the cybersecurity risk.
}

\addnote{Section: Hazardous Incident Prediction}{
    Next, I will elaborate the proposed approach of risk assessment for industrial control systems from two parts:
    \begin{itemize}
      \item hazardous incident prediction
      \item dynamic risk assessment
    \end{itemize}
}

\addnote{Hazardous Incident Prediction: Attack Level}{
    In the proposed approach, the Bayesian network is used to model the relationship between attacks and resources.

    The left figure shows a typical Bayesian network of multi-step attack. In this Bayesian network, the attack nodes, which are colored red, represent attack strategies. the resource nodes, which are color green, represent resources. The enforcement of an attack strategy need some conditions. Only the conditions of an attack strategy is satisfied, may this attack strategy be launched. One the other hands, the enforcement of an attack strategy may obtain another resources. So, using these two kinds of nodes, the Bayesian network can model the multi-step attack.

    The Bayesian network uses the conditional probability table to describe the reachable probability. For example, attack node $a_4$ has two conditions $r_1$ and $r_3$. The first column of the conditional probability table of node $a_4$ shows that when the attacker obtain the resources $r_1$ and $r_3$, the probability that he launches attack $a_4$ is $\ell_{a_4,1}$.\pronunciation{the probability that he launches attack a sub 4 is l sub a sub 4 1} Similarly, if he only has resource $r_1$, the probability is $\ell_{a_4,2}$.\pronunciation{Similarly, if he only has resource r sub 1, the probability is l sub a sub 4 2}
}

\addnote{Hazardous Incident Prediction: Function Tree F1}{
    Function Tree Analysis is widely used to analyze the stability of control system, a typical function tree is shown in following figure.

    If the relationship amongst the functions $f_1$, $f_2$, $f_3$, $f_4$ and $f_5$ is $F_1 = F_2F_3F_4F_5$.\pronunciation{If the relationship amongst the functions lowercase f sub 1, f sub 2, f sub 3, f sub 4 and f sub 5 is capital F sub 1 is equal to capital F sub 2 F sub 3 F sub 4 F sub 5} In this slide, there are two kinds of letter \texttt{F}, where the lowercase \texttt{f} represents the system function, the capital \texttt{F} represents the status of system function $f$. For example, the $F_1$ is equal to \texttt{True} means that the corresponding system function $f_1$ is running normally, the $F_1$ is equal to \texttt{False} means that there is something wrong with the corresponding system function $f_1$.

    Let's go back to the relationship amongst the functions $f_1$, $f_2$, $f_3$, $f_4$ and $f_5$, if the relationship amongst the functions $f_1$, $f_2$, $f_3$, $f_4$ and $f_5$ is $F_1= F_2F_3F_4F_5$. The function tree uses an and-gate to describe this relationship.
}

\addnote{Hazardous Incident Prediction: Function Tree F2}{
    If the relationship amongst the functions $f_2$, $f_6$, $f_7$, $f_8$ and $f_9$ is $F_2 =F_6F_7\overline{F_8}F_9$.\pronunciation{If the relationship amongst the functions lowercase f sub 2, f sub 6, f sub 7, f sub 8 and f sub 9 is capital F sub 2 is equal to capital F sub 6 F sub 7 F sub 8 bar F sub 9} The function tree will uses an appropriate logical gate to describe this kind of relationship.
}

\addnote{Hazardous Incident Prediction: Function Tree F5}{
    Similarly, if the relationship amongst the functions $f_5$, $f_{10}$, $f_{11}$, $f_{12}$ and $f_{13}$ is $F_5 = F_{10} + F_{11} + F_{12} + F_{13}$.\pronunciation{Similarly, if the relationship amongst the functions lowercase f sub 5, f sub 10, f sub 11, f sub 12 and f sub 13 is capital F sub 5 is equal to capital F sub 10 plus F sub 11 plus F sub 12 plus F sub 13} The function tree will uses an or-gate to describe this kind of relationship.
}

\addnote{Hazardous Incident Prediction: Function Tree to Bayesian Network Gate 1}{
    To simplify the inference, the function tree is converted into the Bayesian network, which is shown in following figure.

    This and gate can be converted to a Bayesian network, in which $f_2$, $f_3$, $f_4$ and $f_5$ is the parent nodes of $f_1$. Of cause, a conditional probability table is needed, too.
}

\addnote{Hazardous Incident Prediction: Function Tree to Bayesian Network Gate 2}{
    This kinds of gate can be also converted into a Bayesian network, but the conditional probability table is different. In fact, all kinds of logical gates can be converted into corresponding Bayesian networks.
}

\addnote{Hazardous Incident Prediction: Function Tree to Bayesian Network Gate 3}{
    For example, the or-gate can be converted into the following Bayesian network.
}

\addnote{Hazardous Incident Prediction: Comparison of Fault Tree and Bayesian Network Question}{
    Now, let me digress for a moment. There is a question: can the Bayesan network be converted into the function tree?

    The answer is YES, but not all the Bayesian networks can be converted into the corresponding function trees.

    For example, the following conditional probability table can't be converted into a function tree.
}

\addnote{Hazardous Incident Prediction: Comparison of Fault Tree and Bayesian Network Answer}{
    Because the conditional probability table of the Bayesian network contains more information than the logical gate of the fault tree. In other words, the logical gate cannot always accurately describe the relationship amongst functions.

    the following conditional probability is an example.
}

\addnote{Hazardous Incident Prediction: Incident Level}{
    The occurrence of one incident may cause another incidents, in the proposed approach, the Bayesian network is also used to model the causal relationship amongst the potential incidents. A typical Bayesian network of incident is shown in following figure.

    Like the attack level, the incident node also needs a conditional probability table to describe the relationship amongst it and its parent nodes.
}

\addnote{Hazardous Incident Prediction: Information Transfer from Attack to Function}{
    The attack level, the function level and the incident level have been introduced. Now let's talk about the information transfer between levels.

    The cyber attacks can lead to system function failures, and the function failures may cause the industrial incidents. To analyze the risk propagation, an information transfer is necessary amongst the three aforementioned layers.

    The following figures show the information transfer between attack level and function level.

    We can see that the function failure can lead to another function failures, the launch of attack can also cause function failures. Therefore, we need add some parent nodes for function node $f$. Additional, the conditional probability table of function node $f$ need to be extended.
}

\addnote{Hazardous Incident Prediction: Information Transfer from Function to Incident}{
    Similarly, the information transfer between function level and incident level is shown in the following figure.
}

\addnote{Hazardous Incident Prediction: Collection of Evidence}{
    There are two kind of evidence need to be collected attack evidence and anomaly evidence.
    \begin{itemize}
      \item The attack evidence contains the attack information, such as attack time, attack type, attack object, and so on. The attack evidence can be obtain by the intrusion detection systems.
      \item The anomaly evidence, contains the information about the anomaly, such as function failure, function restoration, incident occurrence, and so on. The anomaly evidence can be obtained by the monitoring systems.
    \end{itemize}

    For each evidence, there exists a corresponding node in the multi-level Bayesian network. When the intrusion detection system or the monitoring system finds an evidence, the corresponding node will be marked in the multi-level Bayesian network.
}

\addnote{Hazardous Incident Prediction: Calculation of Incident Probability}{
    The right figure shows a typical multi-level Bayesian network. We can see it has three levels: attack level colored red and green, function level colored blue, and incident level colored black.

    Assuming that there are three evidences which are detected. They are $a_1$, $a_6$ and $f_1$. Then the nodes $a_1$, $a_6$ and $f_1$ are marked with red dashed circles.

    Finally, the algorithm named Probability Propagation in Trees of Clusters (PPTC) can calculate the probabilities of all the hazardous incidents.

    Now, all the potential hazardous incidences can be calculated. So in summary, first, we collect the evidences about the systems. The evidences contains attack evidences and anomaly evidences. Then we infer what gonna happen according the evidences and their relationship. We use the Bayesian network to model the relationship amongst the evidences.
}

\addnote{Section: Dynamic Risk Assessment}{
    Now we have obtained the probability of all potential hazardous incidents. Let us talk about another problem, how to calculate the dynamic cybersecurity risk?
}

\addnote{Dynamic Risk Assessment: Decouple of Incident Consequences Step 1}{
    As mentioned before, there may exist overlapping amongst different incident consequences. And the overlapping will cause the error of the cybersecurity risk. So the first thing to do is the decouple of consequences.

    For each incident $e_i$, analyze its consequence and generate a consequence set
    \[
        \bm{c}_i = (c_1, c_2, \cdots, c_n) \text{.}\pronunciation{the set c sub i is equal to the set of c sub 1, c sub 2, dots, c sub n}
    \]

    The meaning of $\bm{c}_i$ is that the occurring of the incident $e_i$ will threaten the elements in consequence set $\bm{c}_i$.

    For example, the incident $e_i$ is an explosion of a reactor, which may cause worker casualties, air pollution, facilities damages, and products loss. The consequence set of $e_i$ is
    \[
    \bm{c}_i = (\text{workers}, \text{air}, \text{facilities}, \text{products})\text{.}\pronunciation{c sub i is equal to the set of workers, air, facilities and products}
    \]
}

\addnote{Dynamic Risk Assessment: Decouple of Incident Consequences Step 2 Conditions}{
    Step 2, generate $\bm{C}' = (\bm{c}'_1, \bm{c}'_2, \cdots, \bm{c}'_{m'})$ based on $\bm{C} = (\bm{c}_1, \bm{c}_2, \cdots, \bm{c}_{m'})$.\pronunciation{Step 2, generate set C prime which is equal to set c prime sub 1, set c prime sub 2, dots, set c prime sub m prime, based on set C is equal to set c sub 1, set c sub 2, dots, set c sub m}

    It is noted that all letter \texttt{C} is in bold type, it means that they are all sets. the lowercase \texttt{c} is the set of consequences, the capital \texttt{C} is the set of lowercase \texttt{c}.

    When we generate the set $\bm{C}'$, there are three conditions that set $\bm{C}'$ must be met:

    First, the completeness:
    \[
        \textstyle\bigcup_{i=1}^m\bm{c}_i = \textstyle\bigcup_{i=1}^{m'}\bm{c}'_i\text{.}\pronunciation{First, the completeness, the union of c sub 1 to c sub m is equal to c prime sub 1 to c prime sub m prime}
    \]
    
    Second, the independence:
    \[
        \forall \bm{c}_i',\bm{c}_j' \in \bm{C}' \text{ : } \bm{c}_i' \cap \bm{c}_j' = \varnothing\text{.}\pronunciation{Second, the independence, for each c prime i and c prime j in capital capital C prime, c prime i intersection c prime j is equal to empty set}
    \]
    
    Third, the traceability:
    \[
        \forall \bm{c}' \in \bm{C}', \exists \bm{c} \in \bm{C} \text{ : } \bm{c}' \subseteq \bm{c}\text{.}\pronunciation{Third, the traceability, for each set c prime in capital C prime, there must be a set c in capital C, which c prime is a subset of or equal to c}
    \]

}

\addnote{Dynamic Risk Assessment: Decouple of Incident Consequences Step 2 Example}{
    For example, there are three hazardous incidents $e_1$, $e_2$, and $e_3$, and their consequences are $c_1$, $c_2$, and $c_3$. From the figure we can see that there are overlapping amongst these three consequences. The decoupled consequences are shown in the right figure.

    Obviously, the set $\bm{C}'$ satisfies these three conditions.
}

\addnote{Dynamic Risk Assessment: Decouple of Incident Consequences Step 3}{
    For each $\bm{c}'_j \in \bm{C}'$, we generate a corresponding auxiliary node $x_j$.\pronunciation{For each c prime sub j in capital C prime, we can generate a corresponding auxiliary node x sub j}
    
    According to the traceability of $\bm{C}'$, there must be a consequence set $\bm{c}_i \in \bm{C}$, where $\bm{c}'_j \subseteq \bm{c}_i$.\pronunciation{According to the traceability of capital C prime, there must be a consequences set c sub i in capital C, where c prime sub j is a subset of or equal to c sub i.}

    So, for each $\bm{c}'_j \in \bm{C}'$, we can find the incident set
    \[
    \bm{e}_j = (e_{i_1},e_{i_2},\cdots,e_{i_n})\text{.}\pronunciation{So, for each c prime sub j in capital prime, we can fine the incident set e sub j is equal to e sub i sub 1, e sub i sub 2, dots, e sub i sub n}
    \]
    
    For each incident $e_k$ of the incident set $\bm{e}_j$, the corresponding consequence set $\bm{c}_k$ satisfies the following condition: $\bm{c}'_j \subseteq \bm{c}_k$.\pronunciation{For each incident e sub k of the incident set e sub j , the corresponding consequence set c sub k satisfies the following condition: c prime j is a subset of or equal to c sub k, where c prime sub j is an element of capital C prime}
    
    Therefore, the parent nodes of the auxiliary node $x_j$ are incident nodes $e_{i_1},e_{i_2},\cdots,e_{i_n}$.\pronunciation{Therefore, the parent nodes of the auxiliary node x sub j are incident nodes e sub i sub 1, e sub i sub 2, dots, e sub i sub n}
}

\addnote{Dynamic Risk Assessment: Decouple of Incident Consequences Step 4}{
    The last step, For each auxiliary node x sub j , generate a conditional probability table. A typical conditional probability table of auxiliary node x sub j is shown as following table.

    $H(x_j)$ is equal to 0 only when all its parent nodes are \texttt{False}. In other words, if an incident happens, the auxiliary node will be damaged.

    Now, the consequences has been decoupled. The completeness of set $\bm{C}'$ ensures that there is no omission, and the independence of set $\bm{C}'$ ensures that there is no overlapping.
}

\addnote{Dynamic Risk Assessment: Classification of Incident Consequences}{
    Next, we will talk about the quantification of incident consequences.

    Before the quantification of incident consequences, the adverse effects of an incident may be classified, because different adverse effects need different quantification methods.

    For industrial control systems, the adverse effects of an incident may be classified into three categories: harm to humans, environmental pollution, and property loss.
}

\addnote{Dynamic Risk Assessment: Quantification of Incident Consequences}{
    Different loss of different incident consequences need to be accumulated into the cybersecurity risk, so the loss of incident need to be quantified into same unit. In the proposed approach, we quantify the different loss into monetary loss.

    For the harm to humans, if the decision-maker would like to increase the cost of an investment by $\Delta c$ to reduce the probability of a fatality by $\Delta p$, in proposed approach, the quantification of harm to humans is defined as $Q_H = \Delta c/\Delta p$.\pronunciation{For the harm to humans, if the decision-maker would like to increase the cost of an investment by Delta c to reduce the probability of a fatality by Delta p, in proposed approach, the quantification of harm to humans is defined as Q sub H is equal Delta c over Delta p}

    The monetary loss of environmental pollution is defined as the sum of penalty, compensation, and harness cost.
    \begin{itemize}
      \item  According to the environmental protection laws, if the occurrence of an incident causes environmental pollution, as the owner of industrial control system, the company must pay the penalty charge
      \item When environmental pollution occurs, it tends to influence the living conditions of residents near the plant, the downstream agricultural production, etc. As the relevant liable person, the company has the obligation to pay for compensation.
      \item To clear the polluted environment, as the polluter, the company must take action to improve the environment.
    \end{itemize}
    The monetary loss of property is defined as the cost of replacement.
}

\addnote{Dynamic Risk Assessment: Calculation of Dynamic Risk}{
    Due to the following two reasons:
    \begin{itemize}
      \item there is no overlapping between the consequences of any two auxiliary nodes $x_i$ and $x_j$, $i\neq j$, \pronunciation{i is not equal to j}
      \item the auxiliary nodes contain all the consequences of incidents,
    \end{itemize}
    the dynamic cybersecurity risk can be defined as the following formula:
    \[
        \risk = \sum_{i=1}^{m'}p(x_i)q(x_i)\text{,}\pronunciation{the risk is equal to the sum from i equals one to m prime p x sub i multiplied by q x sub i}
    \]
    where
    \begin{itemize}
      \item $p(x_i)$ is the occurrence probability of the auxiliary node $x_i$,\pronunciation{p x sub i is the occurrence probability of the auxiliary node x sub i}
      \item $q(x_i)$ is the monetary loss of the auxiliary node $x_i$.\pronunciation{q x sub i is the monetary loss of the auxiliary node x sub i}
    \end{itemize}
}

\addnote{Section: Simulation}{
     To illustrate how our approach validly calculates the cybersecurity risk in real-time, a numerical simulation is carried out.
}

\addnote{Simulation: Control Structure of Chemical Reactor}{
    A chemical reactor is a device for containing and controlling a chemical reaction and is widely used in the chemical industry. The representative structure of a chemical reactor control system is shown as the figure.

    In this figure, the Ethernet connects to the enterprise network via gateway 1, which is not shown in this figure. Two CANBUS networks connect to the Ethernet via gateway 2 and gateway 3. In the Ethernet, there are an engineer station and a historical data server. The host in the enterprise network can access the historical data of the historical data server, but cannot access the engineer station. PLC1-6 are distributed into two CANBUS networks. The engineer station and the historical data server can obtain data from all of the PLCs, but only the engineer station can modify and configure PLCs.
}

\addnote{Simulation: Structure of Simulation Platform}{
    This figure shows the structure of the simulation platform.

    The simulation platform is implemented in Matlab, which consists of three modules: an evidence generator, an incident prediction module, and a risk assessment module.

    The evidence generator is used to simulate the intrusion detection system and monitoring system. It uses an array to store an evidence list, which will be shown later. It has a time trigger. To the scheduled time, it will generate an evidence and send it to the incident prediction module.
}

\addnote{Simulation: Multi-Level Bayesian Network of Reactor}{
    This figure shows the multi-Level Bayesian network of reactor.
}

\addnote{Simulation: Evidences List}{
    This table shows the evidence list of the evidence generator.
}

\addnote{Simulation: Quantification of Consequences}{
    This table shows the quantifications of different consequences.
}

\addnote{Simulation: Curvers of Cybersecurity Risk and Incident Probability}{
    
}

\addnote{Simulation: Ability to Deal with the Unknown Attacks}{

}

\addnote{Simulation: Curve of Execution Time Distribution}{

}

\addnote{Simulation: Curves of Risk Assessment Algorithm Scalability}{

}

\addnote{Section: Conclusion and Prospect}{

}

\addnote{Conclusion and Prospect: Conclusion}{

}

\addnote{Conclusion and Prospect: Prospect}{

}

\addnote{Section: Thank You}{

}

\addnote{Thank You: Thank You}{

}

\addnote{Section: Questions}{

}



\end{document}
