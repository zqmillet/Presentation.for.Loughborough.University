\label{Section: Hazardous Incident Prediction}
\section{Hazardous Incident Prediction}
\subsection{The Bayesian Network Based Knowledge Modeling}
\begin{frame}{Attack Level}
    \label{Hazardous Incident Prediction: Attack Level}
    In this paper, the Bayesian network is used to model the relationship between attacks and resources.
    \begin{center}
      \begin{tikzpicture}[line width = \pgfdefaultlinewidth,
                    attack/.style = {draw = red, fill = red, text = white, circle, font = \scriptsize, minimum size = 0.6cm, inner sep = 0pt},
                    resource/.style = {draw = green, fill = green, text = white, circle, font = \scriptsize, minimum size = 0.6cm, inner sep = 0pt}]

    \pgfmathsetmacro{\interval}{2}

    
    \node[attack] (a1) at (1, 0*\interval) {$\bm{a_1}$};
    \node[attack] (a2) at (3, 0*\interval) {$\bm{a_2}$};
    \pause

    \node[resource] (r1) at (1, 1*\interval) {$\bm{r_1}$};
    \node[resource] (r2) at (2, 1*\interval) {$\bm{r_2}$};
    \node[resource] (r3) at (3, 1*\interval) {$\bm{r_3}$};

    \foreach \i/\j in{1/1,
                      1/2,
                      2/1,
                      2/2,
                      2/3}
    {
        \draw[->, red] (a\i) -- (r\j);
    }
    \pause


    \node[attack] (a3) at (1, 2*\interval) {$\bm{a_3}$};
    \node[attack] (a4) at (3, 2*\interval) {$\bm{a_4}$};

    \foreach \i/\j in {1/3,
                       1/4,
                       2/3,
                       3/4}
    {
        \draw[->, green] (r\i) -- (a\j);
    }
    \pause

    \extrarowsep =1mm
    \node[anchor = north west] (CPTR2) at (4,2) {\scriptsize
        \begin{tabu}to 7cm{X[l,-1]@{}|@{}*4{X[c,-1]@{}}}

            $L(a_1)$ & T & T & F & F\\
            $L(a_2)$ & T & F & T & F \\
            \hline
            $O(r_2)$ & $o_{r_2,1}$ & $o_{r_2,2}$ & $o_{r_2,3}$ & $o_{r_2,4}$\\
            $\overline{O}(r_2)$ & $1-o_{r_2,1}$ & $1-o_{r_2,2}$ & $1-o_{r_2,3}$ & $1-o_{r_2,4}$
        \end{tabu}
    };
    \draw[->, dotted] (r2) to[out=-45, in=180] (CPTR2);
    \pause

    \node[anchor = north west] (CPTA4) at (4,4.5) {\scriptsize
        \begin{tabu}to 7cm{X[l,-1]@{}|@{}*4{X[c,-1]@{}}}

            $O(r_1)$ & T & T & F & F\\
            $O(r_3)$ & T & F & T & F\\
            \hline
            $L(a_4)$ & $\ell_{a_4,1}$ & $\ell_{a_4,2}$ & $\ell_{a_4,3}$ & $\ell_{a_4,4}$\\
            $\overline{L}(a_4)$ & $1-\ell_{a_4,1}$ & $1-\ell_{a_4,2}$ & $1-\ell_{a_4,3}$ & $1-\ell_{a_4,4}$
        \end{tabu}
    };
    \draw[->, dotted] (a4) to[out=-45, in=180] (CPTA4);


    \onslide<1->
\end{tikzpicture} 
    \end{center}
\end{frame}

\begin{frame}{Function Level}
    \label<trans:1>{Hazardous Incident Prediction: Function Tree F1}
    \label<trans:2>{Hazardous Incident Prediction: Function Tree F2}
    \label<trans:3>{Hazardous Incident Prediction: Function Tree F5}
    Function Tree Analysis is widely used to analyze the stability of control system, a typical function tree is shown in following figure.
    \begin{center}
      \begin{tikzpicture}[line width = \pgfdefaultlinewidth,
                    circuit logic US,
                    function/.style = {draw = blue, fill = blue, text = white, circle, font = \scriptsize, minimum size = 0.6cm, inner sep = 0pt}]
    \pause
    \node[function] (f1)  at (3.5,4) {$\bm{f_{1}}$};
    \pause
    \pause

    \node[and gate,point up,logic gate inputs=1234,logic gate inverted radius=1pt] (gate1) at (3.5,3) {};
    \draw (gate1.output) -- (f1);

    \node[function] (f2)  at (1.5,2) {$\bm{f_2}$};
    \node[function] (f3) at (1.5+4/3,2) {$\bm{f_{3}}$};
    \node[function] (f4) at (1.5+8/3,2) {$\bm{f_{4}}$};
    \node[function] (f5) at (5.5,2) {$\bm{f_{5}}$};

    \draw (f2.north) -- ++ (up:2mm) -| (gate1.input 1);
    \draw (f3.north) -- ++ (up:1mm) -| (gate1.input 2);
    \draw (f4.north) -- ++ (up:1mm) -| (gate1.input 3);
    \draw (f5.north) -- ++ (up:2mm) -| (gate1.input 4);

    \pause
    \pause

    \node[and gate,point up,logic gate inputs=nnin, logic gate inverted radius=1pt] (gate2) at (1.5,1) {};
    \draw (gate2.output) -- (f2);

    \node[function] (f6) at (0,0) {$\bm{f_6}$};
    \node[function] (f7) at (1,0) {$\bm{f_7}$};
    \node[function] (f8) at (2,0) {$\bm{f_8}$};
    \node[function] (f9) at (3,0) {$\bm{f_9}$};

    \draw (f6.north) -- ++ (up:2mm) -| (gate2.input 1);
    \draw (f7.north) -- ++ (up:1mm) -| (gate2.input 2);
    \draw (f8.north) -- ++ (up:1mm) -| (gate2.input 3);
    \draw (f9.north) -- ++ (up:2mm) -| (gate2.input 4);

    \pause
    \pause

    \node[or gate,point up,logic gate inputs=1234,logic gate inverted radius=1pt] (gate3) at (5.5,1) {};
    \draw (gate3.output) -- (f5);

    \node[function] (f10) at (4,0) {$\bm{f_{10}}$};
    \node[function] (f11) at (5,0) {$\bm{f_{11}}$};
    \node[function] (f12) at (6,0) {$\bm{f_{12}}$};
    \node[function] (f13) at (7,0) {$\bm{f_{13}}$};

    \draw (f10.north) -- ++ (up:2mm) -| (gate3.input 1);
    \draw (f11.north) -- ++ (up:1mm) -| (gate3.input 2);
    \draw (f12.north) -- ++ (up:1mm) -| (gate3.input 3);
    \draw (f13.north) -- ++ (up:2mm) -| (gate3.input 4);

    \onslide<1->
\end{tikzpicture} 
    \end{center}

    \begin{overlayarea}{\textwidth}{1cm}
    \only<3-4| trans:1>{
        \[
            F_1 = F_2F_3F_4F_5
        \]
    }
    \only<5-6| trans:2>{
        \[
            F_2 = F_6F_7\overline{F_8}F_9
        \]
    }
    \only<7-8| trans:3>{
        \[
            F_5 = F_{10}+F_{11}+F_{12}+F_{13}
        \]
    }
    \end{overlayarea}
\end{frame}

\begin{frame}{Function Level}
    \label<trans:1>{Hazardous Incident Prediction: Function Tree to Bayesian Network Gate 1}
    \label<trans:2>{Hazardous Incident Prediction: Function Tree to Bayesian Network Gate 2}
    \label<trans:3>{Hazardous Incident Prediction: Function Tree to Bayesian Network Gate 3}
    To simplify the inference, the function tree is converted into Bayesian network, which is shown in following figure.\\
    \begin{overlayarea}{\textwidth}{6cm}
    \begin{center}
      \only<1-3| trans:1>{
        \begin{tikzpicture}[line width = \pgfdefaultlinewidth,
                    circuit logic US,
                    function/.style = {draw = blue, fill = blue, text = white, circle, font = \scriptsize, minimum size = 0.6cm, inner sep = 0pt}]

    \node[function] (f1) at (2.5,4) {$\bm{f_{1}}$};

    \node[and gate,point up,logic gate inputs=1234,logic gate inverted radius=1pt] (gate1) at (2.5,3) {};
    \draw (gate1.output) -- (f1);

    \node[function] (f2) at (1,2) {$\bm{f_2}$};
    \node[function] (f3) at (2,2) {$\bm{f_{3}}$};
    \node[function] (f4) at (3,2) {$\bm{f_{4}}$};
    \node[function] (f5) at (4,2) {$\bm{f_{5}}$};

    \draw (f2.north) -- ++ (up:2mm) -| (gate1.input 1);
    \draw (f3.north) -- ++ (up:1mm) -| (gate1.input 2);
    \draw (f4.north) -- ++ (up:1mm) -| (gate1.input 3);
    \draw (f5.north) -- ++ (up:2mm) -| (gate1.input 4);


    \node[function] (f1') at (9.5,4) {$\bm{f_{1}}$};

    \node[function] (f2') at (8,2) {$\bm{f_2}$};
    \node[function] (f3') at (9,2) {$\bm{f_3}$};
    \node[function] (f4') at (10,2) {$\bm{f_4}$};
    \node[function] (f5') at (11,2) {$\bm{f_5}$};

    \path[draw=black!50,solid,line width=9mm,fill=black!50, preaction={-triangle 60,line width = 4mm,draw = black!50,shorten >=-10mm}] (4.3,3) -- (7,3);

    \draw[->] (f2') -- (f1');
    \draw[->] (f3') -- (f1');
    \draw[->] (f4') -- (f1');
    \draw[->] (f5') -- (f1');

    \extrarowsep =1mm
    \node[anchor = north] (CPT) at (6,1.5) {\scriptsize
        \begin{tabu}{X[4,c]|*{16}{X[c]}}
          $F(f_2)$ & T & T & T & T & T & T & T & T & F & F & F & F & F & F & F & F \\
          $F(f_3)$ & T & T & T & T & F & F & F & F & T & T & T & T & F & F & F & F \\
          $F(f_4)$ & T & T & F & F & T & T & F & F & T & T & F & F & T & T & F & F \\
          $F(f_5)$ & T & F & T & F & T & F & T & F & T & F & T & F & T & F & T & F \\
          \hline
          $F(f_1)$ & $1$ & $0$ & $0$ & $0$ & $0$ & $0$ & $0$ & $0$ & $0$ & $0$ & $0$ & $0$ & $0$ & $0$ & $0$ & $0$ \\
          $\overline{F}(f_1)$ & $0$ & $1$ & $1$ & $1$ & $1$ & $1$ & $1$ & $1$ & $1$ & $1$ & $1$ & $1$ & $1$ & $1$ & $1$ & $1$
        \end{tabu}
    };

    \onslide<1->
\end{tikzpicture} 
      }
      \only<4-6| trans:2>{
        \begin{tikzpicture}[line width = \pgfdefaultlinewidth,
                    circuit logic US,
                    function/.style = {draw = blue, fill = blue, text = white, circle, font = \scriptsize, minimum size = 0.6cm, inner sep = 0pt}]

    \node[function] (f2) at (2.5,4) {$\bm{f_{2}}$};

    \node[and gate,point up,logic gate inputs=nnin,logic gate inverted radius=1pt] (gate1) at (2.5,3) {};
    \draw (gate1.output) -- (f1);

    \node[function] (f6) at (1,2) {$\bm{f_6}$};
    \node[function] (f7) at (2,2) {$\bm{f_{7}}$};
    \node[function] (f8) at (3,2) {$\bm{f_{8}}$};
    \node[function] (f9) at (4,2) {$\bm{f_{9}}$};

    \draw (f6.north) -- ++ (up:2mm) -| (gate1.input 1);
    \draw (f7.north) -- ++ (up:1mm) -| (gate1.input 2);
    \draw (f8.north) -- ++ (up:1mm) -| (gate1.input 3);
    \draw (f9.north) -- ++ (up:2mm) -| (gate1.input 4);
    \pause\pause\pause\pause

    \path[draw=black!50,solid,line width=9mm,fill=black!50, preaction={-triangle 60,line width = 4mm,draw = black!50,shorten >=-10mm}] (4.3,3) -- (7,3);
    \node[text shadow={[align=center,text width=3cm] at (5.7,3) {Convert}}] at (5.7,3) {Convert};

    \node[function] (f2') at (9.5,4) {$\bm{f_{2}}$};

    \node[function] (f6') at (8,2) {$\bm{f_6}$};
    \node[function] (f7') at (9,2) {$\bm{f_7}$};
    \node[function] (f8') at (10,2) {$\bm{f_8}$};
    \node[function] (f9') at (11,2) {$\bm{f_9}$};

    \draw[->] (f6') -- (f2');
    \draw[->] (f7') -- (f2');
    \draw[->] (f8') -- (f2');
    \draw[->] (f9') -- (f2');
    \pause

    \extrarowsep =1mm
    \node[anchor = north] (CPT) at (6,1.5) {\scriptsize
        \begin{tabu}{X[4,c]|*{16}{X[c]}}
          $F(f_6)$ & T & T & T & T & T & T & T & T & F & F & F & F & F & F & F & F \\
          $F(f_7)$ & T & T & T & T & F & F & F & F & T & T & T & T & F & F & F & F \\
          $F(f_8)$ & T & T & F & F & T & T & F & F & T & T & F & F & T & T & F & F \\
          $F(f_9)$ & T & F & T & F & T & F & T & F & T & F & T & F & T & F & T & F \\
          \hline
          $F(f_1)$ & $1$ & $1$ & $0$ & $1$ & $1$ & $1$ & $1$ & $1$ & $1$ & $1$ & $1$ & $1$ & $1$ & $1$ & $1$ & $1$ \\
          $\overline{F}(f_1)$ & $0$ & $0$ & $1$ & $0$ & $0$ & $0$ & $0$ & $0$ & $0$ & $0$ & $0$ & $0$ & $0$ & $0$ & $0$ & $0$
        \end{tabu}
    };

    \onslide<1->
\end{tikzpicture} 
      }
      \only<7-9| trans:3>{
        \begin{tikzpicture}[line width = \pgfdefaultlinewidth,
                    circuit logic US,
                    function/.style = {draw = blue, fill = blue, text = white, circle, font = \scriptsize, minimum size = 0.6cm, inner sep = 0pt}]

    \node[function] (f5) at (2.5,4) {$\bm{f_{5}}$};

    \node[or gate,point up,logic gate inputs=nnnn,logic gate inverted radius=1pt] (gate1) at (2.5,3) {};
    \draw (gate1.output) -- (f1);

    \node[function] (f10) at (1,2) {$\bm{f_{10}}$};
    \node[function] (f11) at (2,2) {$\bm{f_{11}}$};
    \node[function] (f12) at (3,2) {$\bm{f_{12}}$};
    \node[function] (f13) at (4,2) {$\bm{f_{13}}$};

    \draw (f10.north) -- ++ (up:2mm) -| (gate1.input 1);
    \draw (f11.north) -- ++ (up:1mm) -| (gate1.input 2);
    \draw (f12.north) -- ++ (up:1mm) -| (gate1.input 3);
    \draw (f13.north) -- ++ (up:2mm) -| (gate1.input 4);
    \pause\pause\pause\pause\pause\pause\pause
    
    \path[draw=black!50,solid,line width=9mm,fill=black!50, preaction={-triangle 60,line width = 4mm,draw = black!50,shorten >=-10mm}] (4.3,3) -- (7,3);
    \node[text shadow={[align=center,text width=3cm] at (5.7,3) {Convert}}] at (5.7,3) {Convert};
    \node[function] (f5') at (9.5,4) {$\bm{f_{5}}$};

    \node[function] (f10') at (8,2) {$\bm{f_{10}}$};
    \node[function] (f11') at (9,2) {$\bm{f_{11}}$};
    \node[function] (f12') at (10,2) {$\bm{f_{12}}$};
    \node[function] (f13') at (11,2) {$\bm{f_{13}}$};

    \draw[->] (f10') -- (f5');
    \draw[->] (f11') -- (f5');
    \draw[->] (f12') -- (f5');
    \draw[->] (f13') -- (f5');
    \pause
    
    \extrarowsep =1mm
    \node[anchor = north] (CPT) at (6,1.5) {\scriptsize
        \begin{tabu}{X[4,c]|*{16}{X[c]}}
          $F(f_{10})$ & T & T & T & T & T & T & T & T & F & F & F & F & F & F & F & F \\
          $F(f_{11})$ & T & T & T & T & F & F & F & F & T & T & T & T & F & F & F & F \\
          $F(f_{12})$ & T & T & F & F & T & T & F & F & T & T & F & F & T & T & F & F \\
          $F(f_{13})$ & T & F & T & F & T & F & T & F & T & F & T & F & T & F & T & F \\
          \hline
          $F(f_5)$ & $1$ & $0$ & $0$ & $0$ & $0$ & $0$ & $0$ & $0$ & $0$ & $0$ & $0$ & $0$ & $0$ & $0$ & $0$ & $0$ \\
          $\overline{F}(f_5)$ & $0$  & $1$ & $1$ & $1$ & $1$ & $1$ & $1$ & $1$ & $1$ & $1$ & $1$ & $1$ & $1$ & $1$& $1$ & $1$
        \end{tabu}
    };
    \onslide<1->
\end{tikzpicture} 
      }
    \end{center}
    \end{overlayarea}
\end{frame}

\begin{frame}{Function Level}
    \label<trans:1>{Hazardous Incident Prediction: Comparison of Fault Tree and Bayesian Network Question}
    \label<trans:2>{Hazardous Incident Prediction: Comparison of Fault Tree and Bayesian Network Answer}
    \uncover<4| trans:2>{The conditional probability table of the Bayesian network contains more information than the logical gate of the fault tree.\\}
    \begin{overlayarea}{\textwidth}{6cm}
    \begin{center}
      \begin{tikzpicture}[line width = \pgfdefaultlinewidth,
                    circuit logic US,
                    function/.style = {draw = blue, fill = blue, text = white, circle, font = \scriptsize, minimum size = 0.6cm, inner sep = 0pt}]
    \node[function] (f1) at (9.5,4) {$\bm{f_{1}}$};

    \node[and gate,point up,logic gate inputs=1234,logic gate inverted radius=1pt] (gate1) at (9.5,3) {};
    \draw (gate1.output) -- (f1);

    \node[function] (f2) at (8,2) {$\bm{f_2}$};
    \node[function] (f3) at (9,2) {$\bm{f_{3}}$};
    \node[function] (f4) at (10,2) {$\bm{f_{4}}$};
    \node[function] (f5) at (11,2) {$\bm{f_{5}}$};

    \draw (f2.north) -- ++ (up:2mm) -| (gate1.input 1);
    \draw (f3.north) -- ++ (up:1mm) -| (gate1.input 2);
    \draw (f4.north) -- ++ (up:1mm) -| (gate1.input 3);
    \draw (f5.north) -- ++ (up:2mm) -| (gate1.input 4);

    \path[draw=black!50,solid,line width=9mm,fill=black!50, preaction={-triangle 60,line width = 4mm,draw = black!50,shorten >=-10mm}] (4.3,3) -- (7,3);
    \node[text shadow={[align=center,text width=3cm] at (5.7,3) {Convert}}] at (5.7,3) {Convert};

    \only<1-2| trans:1>{
        \node[text = red] at (5.7,3) {\scalebox{7}{\Black ?}};
    }
    \only<3-| trans:2>{
        \node[forbidden sign,text width=1.5cm, text centered, draw = red, line width = 8pt] at (5.7,3) {};
    }

    \node[function] (f1') at (2.5,4) {$\bm{f_{1}}$};

    \node[function] (f2') at (1,2) {$\bm{f_2}$};
    \node[function] (f3') at (2,2) {$\bm{f_3}$};
    \node[function] (f4') at (3,2) {$\bm{f_4}$};
    \node[function] (f5') at (4,2) {$\bm{f_5}$};

    \draw[->] (f2') -- (f1');
    \draw[->] (f3') -- (f1');
    \draw[->] (f4') -- (f1');
    \draw[->] (f5') -- (f1');
    \pause

    \extrarowsep =1mm
    \node[anchor = north] (CPT) at (6,1.5) {\scriptsize
        \begin{tabu}{X[4,c]|*{16}{X[c]}}
          $F(f_2)$ & T & T & T & T & T & T & T & T & F & F & F & F & F & F & F & \textcolor{red}{\textbf{F}} \\
          $F(f_3)$ & T & T & T & T & F & F & F & F & T & T & T & T & F & F & F & \textcolor{red}{\textbf{F}} \\
          $F(f_4)$ & T & T & F & F & T & T & F & F & T & T & F & F & T & T & F & \textcolor{red}{\textbf{F}} \\
          $F(f_5)$ & T & F & T & F & T & F & T & F & T & F & T & F & T & F & T & \textcolor{red}{\textbf{F}} \\
          \hline
          $F(f_1)$ & $1$ & $1$ & $1$ & $1$ & $1$ & $1$ & $1$ & $1$ & $1$ & $1$ & $1$ & $1$ & $1$ & $1$ & $1$ & \textcolor{red}{$\mathclap{\bm{0.5}}$} \\
          $\overline{F}(f_1)$ & $0$ & $0$ & $0$ & $0$ & $0$ & $0$ & $0$ & $0$ & $0$ & $0$ & $0$ & $0$ & $0$ & $0$ & $0$ & \textcolor{red}{$\mathclap{\bm{0.5}}$}
        \end{tabu}
    };

    \onslide<1->
\end{tikzpicture} 
    \end{center}
    \end{overlayarea}
\end{frame}

\begin{frame}{Incident Level}\label{Hazardous Incident Prediction: Incident Level}
    The occurrence of one incident may cause another incidents, in this paper, the Bayesian network is also used to model the causal relationship amongst the potential incidents.

    A typical Bayesian network of incident is shown in following figure.

    \begin{center}
      \begin{tikzpicture}[line width = \pgfdefaultlinewidth,
                    incident/.style = {draw = black, fill = black, text = white, circle, font = \scriptsize, minimum size = 0.6cm, inner sep = 0pt}]

    \pgfmathsetmacro{\interval}{1.73}

    \node[incident] (e1)  at (0.0, 0*\interval) {$\bm{e_{1 }}$};
    \node[incident] (e2)  at (1.0, 0*\interval) {$\bm{e_{2 }}$};
    \node[incident] (e3)  at (2.0, 0*\interval) {$\bm{e_{3 }}$};
    \node[incident] (e4)  at (3.0, 0*\interval) {$\bm{e_{4 }}$};
    \pause
    
    \node[incident] (e5)  at (0.0, 1*\interval) {$\bm{e_{5 }}$};
    \node[incident] (e6)  at (1.5, 1*\interval) {$\bm{e_{6 }}$};
    \node[incident] (e7)  at (3.0, 1*\interval) {$\bm{e_{7 }}$};

    \foreach \i/\j in {1/5,
                       1/7,
                       2/6,
                       2/7,
                       3/5,
                       4/6,
                       4/7}
    {
        \draw[->] (e\i) -- (e\j);
    }
    \pause
    
    \node[incident] (e8)  at (0.0, 2*\interval) {$\bm{e_{8 }}$};
    \node[incident] (e9)  at (1.0, 2*\interval) {$\bm{e_{9 }}$};
    \node[incident] (e10) at (2.0, 2*\interval) {$\bm{e_{10}}$};
    \node[incident] (e11) at (3.0, 2*\interval) {$\bm{e_{11}}$};
        
    \foreach \i/\j in {5/9,
                       6/8,
                       6/9,
                       6/11,
                       7/10,
                       7/11}
    {
        \draw[->] (e\i) -- (e\j);
    }    
    \pause
    
    \extrarowsep =1mm
    \node[anchor = north west] (CPTE10) at (3.5,2.2*\interval) {\scriptsize
        \begin{tabu}to 4.2cm{X[-1,c]|X[c]X[c]}
            $H(e_7)$    & T & F\\
            \hline
            $\mathclap{H(e_{10})}$ & $h_{e_{10},1}$ & $h_{e_{10},2}$\\
            $\mathclap{\overline{H}(e_{10})}$ & $1-h_{e_{10},1}$ & $1-h_{e_{10},2}$
        \end{tabu}
    };
    \draw[->, dotted] (e10) to[out=-45, in=180] (CPTE10);
    \pause
    
    \node[anchor = north west] (CPTE6) at (3.5,1.1*\interval) {\scriptsize
        \begin{tabu}to 7cm{X[-1,c]|*4{X[c]}}
            $H(e_2)$    & T & T & F & F\\
            $H(e_4)$    & T & F & T & F\\
            \hline
            $H(e_{6})$ & $h_{e_{6},1}$ & $h_{e_{6},2}$ & $h_{e_{6},3}$ & $h_{e_{6},4}$ \\
            $\overline{H}(e_{6})$ & $1-h_{e_{6},1}$ & $1-h_{e_{6},2}$ & $1-h_{e_{6},3}$ & $1-h_{e_{6},4}$
        \end{tabu}
    };       
    \draw[->, dotted] (e6) to[out=-40, in=180] (CPTE6);
    \onslide<1->
\end{tikzpicture} 
    \end{center}
\end{frame}

\begin{frame}{Information Transfer between Levels}
    \label<trans:1>{Hazardous Incident Prediction: Information Transfer from Attack to Function}
    \label<trans:2>{Hazardous Incident Prediction: Information Transfer from Function to Incident}
    The cyber attacks can lead to system function failures, and the function failures may cause the industrial incidents. To analyze the risk propagation, an information transfer is necessary between the three aforementioned layers.

    The following figures show two kind of information transfer.

    \begin{center}
      \only<1| trans:1>{
        \begin{tikzpicture}[line width = \pgfdefaultlinewidth,
                    attack/.style = {draw = red, fill = red, text = white, circle, font = \scriptsize, minimum size = 0.6cm, inner sep = 0pt},
                    function/.style = {draw = blue, fill = blue, text = white, circle, font = \scriptsize, minimum size = 0.6cm, inner sep = 0pt},
                    incident/.style = {draw = black, fill = black, text = white, circle, font = \scriptsize, minimum size = 0.6cm, inner sep = 0pt},
                    dots/.style = {circle, inner sep = 0pt, minimum size = 0.6cm}]

    \pgfmathsetmacro{\interval}{1.73}

    \node[attack]   (a1) at (1.0,1) {$\bm{a_1}$};
    \node[attack]   (a2) at (2.0,1) {$\bm{a_2}$};
    \node[dots  ]   (d1) at (3.0,1) {$\cdots$};
    \node[attack]   (am) at (4.0,1) {$\bm{a_m}$};

    \node[function] (f1) at (5.0,1) {$\bm{f_1}$};
    \node[function] (f2) at (6.0,1) {$\bm{f_2}$};
    \node[dots    ] (d2) at (7.0,1) {$\cdots$};
    \node[function] (fn) at (8.0,1) {$\bm{f_n}$};
    \node[function] (f) at (4.5,3) {$\bm{f}$};

    \node at (0,1) {};
    \node at (9,1) {};

    \draw[->] (a1) -- (f);
    \draw[->] (a2) -- (f);
    \draw[->] (am) -- (f);
    
    \draw[->] (f1) -- (f);
    \draw[->] (f2) -- (f);
    \draw[->] (fn) -- (f);

    \draw (2.5,0.4) node {$\underbrace{\hspace{3.7cm}}$} (2.5,0.3) node[anchor = north] {$m$ attack nodes};
    \draw (6.5,0.4) node {$\underbrace{\hspace{3.7cm}}$} (6.5,0.3) node[anchor = north] {$n$ function nodes};
    \onslide<1->
\end{tikzpicture} 
      }
      \only<2| trans:2>{
        \begin{tikzpicture}[line width = \pgfdefaultlinewidth,
                    attack/.style = {draw = red, fill = red, text = white, circle, font = \scriptsize, minimum size = 0.6cm, inner sep = 0pt},
                    function/.style = {draw = blue, fill = blue, text = white, circle, font = \scriptsize, minimum size = 0.6cm, inner sep = 0pt},
                    incident/.style = {draw = black, fill = black, text = white, circle, font = \scriptsize, minimum size = 0.6cm, inner sep = 0pt},
                    dots/.style = {circle, inner sep = 0pt, minimum size = 0.6cm}]

    \pgfmathsetmacro{\interval}{1.73}

    \node[function]   (f1) at (1.0,1) {$\bm{f_1}$};
    \node[function]   (f2) at (2.0,1) {$\bm{f_2}$};
    \node[dots  ]     (d1) at (3.0,1) {$\cdots$};
    \node[function]   (fm) at (4.0,1) {$\bm{f_m}$};

    \node[incident]   (e1) at (5.0,1) {$\bm{e_1}$};
    \node[incident]   (e2) at (6.0,1) {$\bm{e_2}$};
    \node[dots    ]   (d2) at (7.0,1) {$\cdots$};
    \node[incident]   (en) at (8.0,1) {$\bm{e_n}$};
    \node[incident]   (e)  at (4.5,3) {$\bm{e}$};
    
    \node at (0,1) {};
    \node at (9,1) {};
    
    \draw[->] (f1) -- (e);
    \draw[->] (f2) -- (e);
    \draw[->] (fm) -- (e);

    \draw[->] (e1) -- (e);
    \draw[->] (e2) -- (e);
    \draw[->] (en) -- (e);

    \draw (2.5,0.4) node {$\underbrace{\hspace{3.7cm}}$} (2.5,0.3) node[anchor = north] {$m$ function nodes};
    \draw (6.5,0.4) node {$\underbrace{\hspace{3.7cm}}$} (6.5,0.3) node[anchor = north] {$n$ incident nodes};
    \onslide<1->
\end{tikzpicture} 
      }
    \end{center}
\end{frame}

\subsection{Incident Prediction}
\begin{frame}{Collection of Evidence}\label{Hazardous Incident Prediction: Collection of Evidence}
    There are two kind of evidence need to be collected:
    \begin{itemize}
      \item \textbf{Attack Evidence}, contains the attack information, such as attack time, attack type, attack object, etc.
      \item \textbf{Anomaly Evidence}, contains the information about the anomaly, such as function failure, function restoration, incident occurrence, etc.
    \end{itemize}
    \pause

    For each evidence, there exists a corresponding node in the multi-level Bayesian network. When the intrusion detection system or the monitoring system finds an evidence, the corresponding node will be marked in the multi-level Bayesian network.
\end{frame}

\begin{frame}{Calculation of Incident Probability}\label{Hazardous Incident Prediction: Calculation of Incident Probability}
    \begin{minipage}[][6cm][t]{0.4\textwidth}
      \begin{tikzpicture}[line width = \pgfdefaultlinewidth,
                    attack/.style = {draw = red, fill = red, text = white, circle, font = \scriptsize, minimum size = 0.6cm, inner sep = 0pt},
                    resource/.style = {draw = green, fill = green, text = white, circle, font = \scriptsize, minimum size = 0.6cm, inner sep = 0pt},
                    function/.style = {draw = blue, fill = blue, text = white, circle, font = \scriptsize, minimum size = 0.6cm, inner sep = 0pt},
                    incident/.style = {draw = black, fill = black, text = white, circle, font = \scriptsize, minimum size = 0.6cm, inner sep = 0pt},
                    dots/.style = {circle, inner sep = 0pt, minimum size = 0.6cm}]

    \pgfmathsetmacro{\interval}{1.2}
    
    \node[attack]   (a1) at (0, 0*\interval) {$\bm{a_1}$};
    \node[attack]   (a2) at (2, 0*\interval) {$\bm{a_2}$};
    \node[attack]   (a3) at (4, 0*\interval) {$\bm{a_3}$};
    \node[attack]   (a4) at (6, 0*\interval) {$\bm{a_4}$};
    
    \node[resource] (r1) at (0, 1*\interval) {$\bm{r_1}$};
    \node[resource] (r2) at (3, 1*\interval) {$\bm{r_2}$};
    \node[resource] (r3) at (6, 1*\interval) {$\bm{r_3}$};
    
    \foreach \i/\j in {1/1,
                       1/2,
                       2/1,
                       2/3,
                       3/1,
                       3/2,
                       4/3}
    {
        \draw[->] (a\i) -- (r\j);
    }
    
    \node[attack]   (a5) at (0, 2*\interval) {$\bm{a_5}$};
    \node[attack]   (a6) at (2, 2*\interval) {$\bm{a_6}$};
    \node[attack]   (a7) at (4, 2*\interval) {$\bm{a_7}$};
    \node[attack]   (a8) at (6, 2*\interval) {$\bm{a_8}$};
    
    \foreach \i/\j in {1/5,
                       1/6,
                       2/6,
                       2/6,
                       3/7,
                       3/8}
    {
        \draw[->] (r\i) -- (a\j);
    }
    
    \node[function] (f1) at (0, 3*\interval) {$\bm{f_1}$};
    \node[function] (f2) at (3, 3*\interval) {$\bm{f_2}$};
    \node[function] (f3) at (6, 3*\interval) {$\bm{f_3}$};
    
    \foreach \i/\j in {5/2,
                       5/3,
                       6/1,
                       7/1,
                       7/2,
                       7/3,
                       8/2,
                       8/3}
    {
        \draw[->] (a\i) -- (f\j);
    }
    
    \node[incident] (e1) at (0, 4*\interval) {$\bm{e_1}$};
    \node[incident] (e2) at (2, 4*\interval) {$\bm{e_2}$};
    \node[incident] (e3) at (4, 4*\interval) {$\bm{e_3}$};
    \node[incident] (e4) at (6, 4*\interval) {$\bm{e_4}$};
    
    \foreach \i/\j in {1/1,
                       1/3,
                       2/1,
                       2/2,
                       3/3,
                       3/4}
    {
        \draw[->] (f\i) -- (e\j);
    }
    
\end{tikzpicture} 
    \end{minipage}
    \begin{minipage}[][6cm][t]{0.6\textwidth}
        \vspace{5pt}The left figure shows a typical multi-level Bayesian network.\\[0pt]

        \pause
        Assuming that the evidence list is
        \[
            a_1, a_6, f_1
        \]
        \pause

        \vspace{-10pt}Then the nodes $a_1$, $a_6$, and $f_1$ are marked with \textcolor{red}{\textbf{red}} dashed circles.\\[0pt]
        \pause

        Finally, the algorithm named Probability Propagation in Trees of Clusters (PPTC) can calculate the probabilities of all the hazardous incidents.
    \end{minipage}
\end{frame} 